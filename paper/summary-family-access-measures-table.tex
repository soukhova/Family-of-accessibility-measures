{\tiny
\begin{longtable}{|p{2.5cm}|p{2.5cm}|p{2.5cm}|p{3cm}|p{3cm}|}
\hline
\textbf{Name of Case and Variant} & \textbf{Constraint Explanation and Balancing Factor} & \textbf{Proportional Allocation Factor} & \textbf{Measure Equation} & \textbf{Interpretation} \\
\hline
\endfirsthead

\hline
\textbf{Name of Case and Variant} & \textbf{Constraint Explanation and Balancing Factor} & \textbf{Proportional Allocation Factor} & \textbf{Measure Equation} & \textbf{Interpretation} \\
\hline
\endhead

Unconstrained Accessible Opportunities ($V_i^0$) and Unconstrained Accessibile Population ($M_j^0$)
& No constraints; marginals not equal to any regional or zonal knowns.
& None
& $V_i^0 = \sum_j O_j f(c_{ij})$;

$M_j^0 = \sum_i P_i f(c_{ij})$

& Values in various units depending on the impedance and destination-mass (e.g., "opportunities x decay") for $V_i^0$ andimpedance and origin-mass (e.g., "population x decay"); no total or marginal constraint \\
\hline

Total Constrained Accessible Opportunities ($V_i^T$) and Total Constrained Accessibile Population ($M_j^T$)
& Balancing factors $K^T$ and $\hat{K}^T$ ensures the sum of $ij$ values equals the total marginal, where:

$K^T = \frac{D}{\sum_i V_i^0}$;

$\hat{K}^T = \frac{O}{\sum_j M_j^0}$

& Allocates the total marginal as opportunities based on $\kappa_i^T = \sum_j K^T f(c_{ij})$ and as population based on  $\hat{\kappa}_j^T = \sum_i \hat K^T f(c_{ij})$

& $V^T_i = \kappa_i^T \sum_j W^{(2)}_j$;

$M_j^T = \hat{\kappa}_j^T \sum_i W^{(1)}_i$


& Values reflect a share of total regional opportunities ($V^T_i$) or population ($M^T_j$). \\
\hline

Singly Constrained Accessible Opportunities ($V_i^S$) and Singly Constrained Accessible Population ($M_j^S$)
& Single balancing factor $B_j$ (for $V_i^S$) that ensures the destination-mass marginal is constrained, and $A_i$ (for $M_j^S$) ensures the origin-mass marginal is constrained:

$B_j = \frac{1}{\sum_i W_i^{(1)} f(c_{ij})}$;

$A_i = \frac{1}{\sum_j W_j^{(2)} f(c_{ij})}$

& Allocates the single opportunities marginal proportionally based on $\kappa^S_i = W_i^{(1)} B_j$ in the case of $V_i^S$ and the single population marginal proroportionally based on $\hat \kappa^S_j = W_j^{(2)} A_i$ in the case of $M_j^S$

& $V^S_i = \kappa^S_i \sum_j D_j$;

$M_j^S = \hat \kappa^S_j \sum_i P_i$

& $V^S_i$ values reflect a share of the opportunities at each destination based on origin population 'demand' and impedance; $M^S_j$ values reflect a share of the population at each origin based on destination opportunities 'supply' and impedance.\\
\hline

Doubly Constrained Access ($V_{ij}^D$ or $M_{ij}^D$)
& Values reflect both single marginals simutaneously, maintained via $A_i$ and $B_j$.
& —
& $V_{ij}^D = A_i B_j P_i O_j f(c_{ij})$
& The spatial interactions between population and opportunities (i.e., access). \\
\hline

\end{longtable}
}
