% Options for packages loaded elsewhere
% Options for packages loaded elsewhere
\PassOptionsToPackage{unicode}{hyperref}
\PassOptionsToPackage{hyphens}{url}
%


\PassOptionsToPackage{table}{xcolor}

\documentclass[
  10pt,
  letterpaper,
]{article}
\usepackage{xcolor}
\usepackage[top=0.85in,left=2.75in,footskip=0.75in]{geometry}
\usepackage{amsmath,amssymb}
\setcounter{secnumdepth}{-\maxdimen} % remove section numbering
\usepackage{iftex}
\ifPDFTeX
  \usepackage[T1]{fontenc}
  \usepackage[utf8]{inputenc}
  \usepackage{textcomp} % provide euro and other symbols
\else % if luatex or xetex
  \usepackage{unicode-math} % this also loads fontspec
  \defaultfontfeatures{Scale=MatchLowercase}
  \defaultfontfeatures[\rmfamily]{Ligatures=TeX,Scale=1}
\fi
\usepackage{lmodern}
\ifPDFTeX\else
  % xetex/luatex font selection
\fi
% Use upquote if available, for straight quotes in verbatim environments
\IfFileExists{upquote.sty}{\usepackage{upquote}}{}
\IfFileExists{microtype.sty}{% use microtype if available
  \usepackage[]{microtype}
  \UseMicrotypeSet[protrusion]{basicmath} % disable protrusion for tt fonts
}{}
\makeatletter
\@ifundefined{KOMAClassName}{% if non-KOMA class
  \IfFileExists{parskip.sty}{%
    \usepackage{parskip}
  }{% else
    \setlength{\parindent}{0pt}
    \setlength{\parskip}{6pt plus 2pt minus 1pt}}
}{% if KOMA class
  \KOMAoptions{parskip=half}}
\makeatother


\usepackage{longtable,booktabs,array}
\usepackage{calc} % for calculating minipage widths
% Correct order of tables after \paragraph or \subparagraph
\usepackage{etoolbox}
\makeatletter
\patchcmd\longtable{\par}{\if@noskipsec\mbox{}\fi\par}{}{}
\makeatother
% Allow footnotes in longtable head/foot
\IfFileExists{footnotehyper.sty}{\usepackage{footnotehyper}}{\usepackage{footnote}}
\makesavenoteenv{longtable}
\usepackage{graphicx}
\makeatletter
\newsavebox\pandoc@box
\newcommand*\pandocbounded[1]{% scales image to fit in text height/width
  \sbox\pandoc@box{#1}%
  \Gscale@div\@tempa{\textheight}{\dimexpr\ht\pandoc@box+\dp\pandoc@box\relax}%
  \Gscale@div\@tempb{\linewidth}{\wd\pandoc@box}%
  \ifdim\@tempb\p@<\@tempa\p@\let\@tempa\@tempb\fi% select the smaller of both
  \ifdim\@tempa\p@<\p@\scalebox{\@tempa}{\usebox\pandoc@box}%
  \else\usebox{\pandoc@box}%
  \fi%
}
% Set default figure placement to htbp
\def\fps@figure{htbp}
\makeatother





\setlength{\emergencystretch}{3em} % prevent overfull lines

\providecommand{\tightlist}{%
  \setlength{\itemsep}{0pt}\setlength{\parskip}{0pt}}



 
\usepackage[numbers,square,comma]{natbib}
\bibliographystyle{plos2015}


% Use adjustwidth environment to exceed column width (see example table in text)
\usepackage{changepage}

% marvosym package for additional characters
\usepackage{marvosym}

% cite package, to clean up citations in the main text. Do not remove.
% Using natbib instead
% \usepackage{cite}

% Use nameref to cite supporting information files (see Supporting Information section for more info)
\usepackage{nameref,hyperref}

% line numbers
\usepackage[right]{lineno}

% ligatures disabled
\usepackage{microtype}
\DisableLigatures[f]{encoding = *, family = * }

% create "+" rule type for thick vertical lines
\newcolumntype{+}{!{\vrule width 2pt}}

% create \thickcline for thick horizontal lines of variable length
\newlength\savedwidth
\newcommand\thickcline[1]{%
  \noalign{\global\savedwidth\arrayrulewidth\global\arrayrulewidth 2pt}%
  \cline{#1}%
  \noalign{\vskip\arrayrulewidth}%
  \noalign{\global\arrayrulewidth\savedwidth}%
}

% \thickhline command for thick horizontal lines that span the table
\newcommand\thickhline{\noalign{\global\savedwidth\arrayrulewidth\global\arrayrulewidth 2pt}%
\hline
\noalign{\global\arrayrulewidth\savedwidth}}

% Text layout
\raggedright
\setlength{\parindent}{0.5cm}
\textwidth 5.25in 
\textheight 8.75in

% Bold the 'Figure #' in the caption and separate it from the title/caption with a period
% Captions will be left justified
\usepackage[aboveskip=1pt,labelfont=bf,labelsep=period,justification=raggedright,singlelinecheck=off]{caption}
\renewcommand{\figurename}{Fig}

% Remove brackets from numbering in List of References
\makeatletter
\renewcommand{\@biblabel}[1]{\quad#1.}
\makeatother

% Header and Footer with logo
\usepackage{lastpage,fancyhdr}
\usepackage{epstopdf}
%\pagestyle{myheadings}
\pagestyle{fancy}
\fancyhf{}
%\setlength{\headheight}{27.023pt}
%\lhead{\includegraphics[width=2.0in]{PLOS-submission.eps}}
\rfoot{\thepage/\pageref{LastPage}}
\renewcommand{\headrulewidth}{0pt}
\renewcommand{\footrule}{\hrule height 2pt \vspace{2mm}}
\fancyheadoffset[L]{2.25in}
\fancyfootoffset[L]{2.25in}
\lfoot{\today}
\usepackage{booktabs}
\usepackage{caption}
\usepackage{longtable}
\usepackage{colortbl}
\usepackage{array}
\usepackage{anyfontsize}
\usepackage{multirow}
\usepackage{float}
\floatplacement{table}{H}
\floatplacement{figure}{H}
\usepackage{pdflscape}
\newcommand{\blandscape}{\begin{landscape}}
\newcommand{\elandscape}{\end{landscape}}
\makeatletter
\@ifpackageloaded{caption}{}{\usepackage{caption}}
\AtBeginDocument{%
\ifdefined\contentsname
  \renewcommand*\contentsname{Table of contents}
\else
  \newcommand\contentsname{Table of contents}
\fi
\ifdefined\listfigurename
  \renewcommand*\listfigurename{List of Figures}
\else
  \newcommand\listfigurename{List of Figures}
\fi
\ifdefined\listtablename
  \renewcommand*\listtablename{List of Tables}
\else
  \newcommand\listtablename{List of Tables}
\fi
\ifdefined\figurename
  \renewcommand*\figurename{Figure}
\else
  \newcommand\figurename{Figure}
\fi
\ifdefined\tablename
  \renewcommand*\tablename{Table}
\else
  \newcommand\tablename{Table}
\fi
}
\@ifpackageloaded{float}{}{\usepackage{float}}
\floatstyle{ruled}
\@ifundefined{c@chapter}{\newfloat{codelisting}{h}{lop}}{\newfloat{codelisting}{h}{lop}[chapter]}
\floatname{codelisting}{Listing}
\newcommand*\listoflistings{\listof{codelisting}{List of Listings}}
\makeatother
\makeatletter
\makeatother
\makeatletter
\@ifpackageloaded{caption}{}{\usepackage{caption}}
\@ifpackageloaded{subcaption}{}{\usepackage{subcaption}}
\makeatother
\usepackage{bookmark}
\IfFileExists{xurl.sty}{\usepackage{xurl}}{} % add URL line breaks if available
\urlstyle{same}
\hypersetup{
  pdftitle={A family of accessibility measures derived from spatial interaction principles},
  pdfauthor={Anastasia Soukhov; Rafael H. M. Pereira; Christopher D. Higgins; Antonio Páez},
  hidelinks,
  pdfcreator={LaTeX via pandoc}}




\begin{document}
\vspace*{0.2in}

% Title must be 250 characters or less.
\begin{flushleft}
{\Large
\textbf\newline{A family of accessibility measures derived from spatial
interaction
principles} % Please use "sentence case" for title and headings (capitalize only the first word in a title (or heading), the first word in a subtitle (or subheading), and any proper nouns).
}
\newline
\\
% Insert author names, affiliations and corresponding author email (do not include titles, positions, or degrees).
Anastasia Soukhov\textsuperscript{1}, Rafael H. M.
Pereira\textsuperscript{2}, Christopher D.
Higgins\textsuperscript{3*}, Antonio Páez\textsuperscript{1}
\\
\bigskip
\textbf{1} McMaster University, School of Earth, Environment \&
Society, Hamilton, Canada, \\ \textbf{2} Institute for Applied Economic
Research - Ipea, Data Science Division, Brasília,
Brazil, \\ \textbf{3} University of Toronto Scarborough, Department of
Human Geography, Toronto, Canada, 
\bigskip

% Insert additional author notes using the symbols described below. Insert symbol callouts after author names as necessary.
% 
% Remove or comment out the author notes below if they aren't used.
%
% Primary Equal Contribution Note
\Yinyang These authors contributed equally to this work.

% Additional Equal Contribution Note
% Also use this double-dagger symbol for special authorship notes, such as senior authorship.
%\ddag These authors also contributed equally to this work.

% Current address notes
\textcurrency Current Address: Dept/Program/Center, Institution Name, City, State, Country % change symbol to "\textcurrency a" if more than one current address note
% \textcurrency b Insert second current address 
% \textcurrency c Insert third current address

% Deceased author note
\dag Deceased

% Group/Consortium Author Note
\textpilcrow Membership list can be found in the Acknowledgments
sections

% Use the asterisk to denote corresponding authorship and provide email address in note below.
* cd.higgins@utoronto.ca

\end{flushleft}

\section*{Abstract}
Transportation planning has long prioritized the efficiency of movement.
However, the concept of accessibility represents a more comprehensive
evolution, shifting focus from movement (i.e., trips) to the potential
to spatially interact with desired destinations. Despite growing
recognition of accessibility-based planning approaches, the concept
remains fragmented, with inconsistent definitions and unclear
interpretations. To this end, this paper makes a methodological
contribution by specifying a family of accessibility measures that are
grounded in the shared `gravity-based' theoretical roots of spatial
interaction models, particularly their balancing factors. From this
foundation, we outline four members of the family: the `unconstrained'
measure (i.e., Hansen-type accessibility), the `total-constrained'
measure (i.e., a constrained version of the Hansen-type accessibility),
the `singly-constrained' measure (i.e., related to the popular two-step
floating catchment approach -- 2SFCA), and the `doubly-constrained'
measure representing realized access (i.e., equal to the
doubly-constrained spatial interaction model). These measures can be
interpreted as either the number of accessible opportunities or
accessible population (i.e., market potential). A toy example
illustrates how they produce interpretable unit-based values, offering a
clearer and more coherent basis for accessibility analysis.


\linenumbers

\section{1. Introduction}\label{introduction}

In the early twentieth century, the emergence of a transportation
planning paradigm focused primarily on mobility cemented major
investments in automobile and transportation infrastructure, fostering
lower-density sprawl, car-dependent development and entrenching
automobility in planning practice
\citep{miller_collaborative_2011, lavery_driving_2013}. Within this new
practice, access to destinations was treated as a by-product of
movement. Despite continued road and highway expansion, this
automobility monoculture has proven ineffective at reducing travel costs
or environmental burdens, and has not clearly improved people's ability
to reach destinations
\citep{farber_running_2011, handyACCESSIBILITYVSMOBILITYENHANCING2002, paez_healthcare_2010}.

In response, transportation researchers have increasingly advocated for
the adoption of accessibility as a planning criterion, in contrast to
traditional mobility-oriented transportation planning approaches which
translate into indicators that benchmark movement (e.g., vehicle
kilometres traveled, intersection through traffic) which are not
necessarily linked to improved accessibility
\citep{silvaAccessibilityInstrumentsPlanning2017, paez_developing_2013, handy2020, elgeneidyMakingAccessibilityWork2022}.
Accessibility, by contrast, is the ``potential of opportunities for
{[}spatial{]} interaction'' \citep{hansen1959}. While mobility reflects
movement, accessibility captures the combined influence of transport and
land use, emphasizing destinations and the potential to reach them
\citep{handyMeasuringAccessibilityExploration1997}.

Accessibility research has expanded across diverse domains including:
employment
\citep{karstEvaluationAccessibilityImpacts2003, grengs2010job, paez_jobs_2013, merlin2017competition, tao_investigating_2020},
healthcare
\citep{luo2003, paez_healthcare_2010, wan2012three, delamater2013spatial, boisjoly2017informality, pereira_2021_geographic, yang2024evaluating},
green space
\citep{reyesAccessibility2014, rojas_accessibility_2016, liang_novel_2024},
education
\citep{williams_disparities_2014, romanillosAccessibilitySchoolsSpatial2018, marques_accessibility_2021},
social contact
\citep{neutens_human_2007, farberActivitySpacesMeasurement2012, farber_2013_social},
and regional economics
\citep{vickermanAccessibility1999, lopezMeasuring2008, ribeiro_road_2010, gutierrez_evaluating_2011},
among many other domains of application. Despite its popularity in
scholarly works, accessibility still remains difficult to implement in
planning due to definitional inconsistencies
\citep{vanweeAccessible2016, handy2020, kapatsila_resolving_2023} and
challenges in interpreting and communicating results
\citep{geursAccessibilityEvaluationLanduse2004, vanweeAccessible2016, ferreiraReenactingMobilityAccessibility2020}.

More specifically, the wide range of accessibility definitions, with
novel methods being more sophisticated but less intuitive
\citep{kapatsila_resolving_2023}, can further hinder practical uptake
\citep{vanweeAccessible2016}. Geurs and van Wee
\citep{geursAccessibilityEvaluationLanduse2004} classify accessibility
measures into four categories: infrastructure-, place-, person-, and
utility-based. Among place-based measures (this work's focus), variants
include gravity-based \citep{hansen1959}, cumulative opportunity
\citep{pirie_measuring_1979}, the 2 Step Floating Catchment Area (FCA)
approach \citep{luo2003}, and a variety of modifications to these
approaches e.g., Enhanced 2-Step FCA \citep{luoEnhanced2009}, 3-Stage
FCA \citep{wan2012three}, Modified 2-Step FCA
\citep{delamater2013spatial}, inverse 2-Step FCA
\citep{wang_2sfca_2021}, and n-steps FCA \citep{liang_novel_2024}. While
these methods are tailored to address specific research contexts,
overall this diversity does not demystify existing questions like those
raised by van Wee \citep{vanweeAccessible2016}: How should practitioners
interpret differences in accessibility scores between modes, and how
should results be communicated?

Rather than propose a new measure, this work argues for a return to the
spatial interaction foundations of accessibility. Specifically, we show
how the family of spatial interaction models \citep{wilson1971} can be
reformulated in the context of accessibility, namely as a ``family of
accessibility measures''. This formulation results in constrained
versions of gravity-based accessibility \citep[e.g.,][]{hansen1959}, but
in units of \emph{spatially reachable opportunities}. This approach
offers a direct mathematical link to existing accessibility measures
while restoring tangible meaning to zonal values. Instead of abstract
proportional scores, constrained accessibility expresses the number of
opportunities a population may potentially spatially interact with.

This paper makes two contributions. First, we review how spatial
interaction modeling and accessibility share similar origins but have
diverged in focus and interpretation. Second, we introduce a family of
accessibility measures grounded in spatial interaction principles,
including total, singly, and doubly-constrained cases and variants of
``accessible opportunities'' and ``accessible population''. These cases
and variants align with common measures such as Hansen-type
accessibility \citep{hansen1959}, competitive accessibility measures
such as the 2SFCA method \citep{shen1998, luo2003}, and market potential
models
\citep{harris_market_1954, vickermanAccessibilityAttractionPotential1974}.

We contend that accessibility research should re-engage with spatial
interaction modeling, particularly the use of Wilson's
\citep{wilson1971} system constraints. While spatial interaction models
embraced such constraints to improve interpretability, most
accessibility models have not. This lack of adoption contributes to
fuzziness in current analyses, limiting interpretive clarity to simple
proportional comparisons (e.g., ``higher-than'', ``lower-than'')
\citep{millerAccessibilityMeasurementApplication2018}. Without such
constraints, accessibility scores lack clear units and comparability
across cities or modes. In contrast, constrained measures yield values
that can be tied to tangible values without any post-hoc treatment,
theoretically making them more interpretable, communicable, and
actionable in planning.

The remainder of this paper proceeds as follows. Sections 2 through 4
trace the historical development of spatial interaction and
accessibility research, beginning with Newtonian gravitational analogies
and Carey (1858) \citep{careyPrinciplesSocialScience1858} (Section 2),
moving through early researchers like Ravenstein (1885)
\citep{ravensteinLawsMigration1885} to Stewart (1948)
\citep{stewartDemographicGravitationEvidence1948} who theorized and
formalized spatial interaction patterns (Section 3), and examining how
the ``gravity-based'' accessibility approach in Hansen (1959)
\citep{hansen1959} became the dominant approach in planning practice
(Section 4). Section 5 presents the entropy-based family of spatial
interaction models in Wilson (1971) \citep{wilson1971}, explaining how
the introduction of ``constraints'' (based on top-down known information
as part of entropy-maximization) produces interpretable, unit-consistent
flow estimates. Section 6 explores why accessibility and spatial
interaction literature diverged into separate branches despite their
shared conceptual foundations. Section 7 constitutes the paper's core
contribution: we derive a family of accessibility measures corresponding
to different constraints applied -- unconstrained, singly-constrained,
and doubly-constrained cases -- demonstrating how each yields zone-level
accessibility values expressed in meaningful units (opportunities or
population) rather than unit-inconsistent indices. We illustrate each
measure with numerical examples that clarify the practical implications
of different constraint assumptions. Section 8 concludes by discussing
how these constrained measures can inform planning decisions and improve
clarity in accessibility analysis.

The key insight is that accessibility need not be unit-inconsistent. By
building from spatial interaction theory's constraint framework rather
than the gravitational analogy alone, we show how to construct
accessibility measures that are simultaneously grounded in behavioural
principles (the gravitational analogy) as well as being expressed in
interpretable units (opportunities or population) that are sensitive to
the known to the region through empirically-set constraints.

\section{2. Newtonian's roots of human spatial interaction
research}\label{newtonians-roots-of-human-spatial-interaction-research}

The patterns of people's movement in space have been a subject of
scientific inquiry for at least a century and a half, from as far back
as Henry C. Carey's \emph{Principles of Social Science}
\citep{careyPrinciplesSocialScience1858}. It was in this work where
Carey stated that ``man {[}is{]} the molecule of society {[}and their
interaction is subject to{]} the direct ratio of the mass and the
inverse one of distance'' \citep[pp.~37-38]{mckeanManual1883}. This
statement shows how investigations into human spatial interaction have
often been explicitly coloured by the features of Newton's Law of
Universal Gravitation, first posited in 1687's \emph{Principia
Mathematica} and expressed as in Eq~\ref{eq-phys-grav-prop}.

\begin{equation}\phantomsection\label{eq-phys-grav-prop}{
F_{ij} \propto \frac{M_i M_j} {D_{ij}^{2}}
}\end{equation}

This famous equation expresses that the attractive force \(F\) between
two bodies \(i\) and \(j\) is directly proportional to the product of
their masses and inversely proportional to the square of the distance
between them. As mass increases, so does force; as distance increases,
force decreases. However, Eq~\ref{eq-phys-grav-prop} only demonstrates a
\emph{proportional} relationship. To quantify the magnitude (not the
\emph{proportional} magnitude) of the force, it must be
\emph{constrained} with an empirical constant. This constant \(G\)
converts Eq~\ref{eq-phys-grav-prop} from an expression of
proportionality to the following expression of equality:

\begin{equation}\phantomsection\label{eq-phys-grav}{
F_{ij} = G \frac{M_i M_j} {D_{ij}^{2}}
}\end{equation}

Where \(G\) is the gravitational constant -- an empirically calibrated
value that ensures the model reflects observed forces. Newton's initial
estimate of \(G\) was based on a speculation but received empirical
support after Hutton's and Cavendish's experiments in the late 1700s
\citep{hutton_xxxiii_1778, cavendish_xxi_1798}, which estimated it to
within 1\% accuracy. That is, it took over a century from the
publication of Newton's \emph{Principia Mathematica} to refine the
estimate of the proportionality constant.

While the Newtonian gravitational relationship laid the conceptual
groundwork for later empirical studies of human spatial interaction, the
majority of these early attempts described a proportional relationship
or one arbitrarily set to equality. They did not establish an empirical
\(G\) as in the Newtownian tradition, as will be discussed in the next
subsection.

\section{3. Early research on human spatial interaction: from Ravenstein
(1885) to Stewart
(1948)}\label{early-research-on-human-spatial-interaction-from-ravenstein-1885-to-stewart-1948}

Henry C. Carey's \emph{Principles of Social Science}
\citep{careyPrinciplesSocialScience1858} inspired empirical spatial
interaction research in different contexts. Namely, a number of
researchers theoretically and empirically attempted to characterize
human spatial interaction as a force \(F\) directly proportional to the
``masses'' \(M_i\) and \(M_j\) of two locations, and inversely
proportional to their separation distance -- conceptually parallel to
Newtonian gravity, but often omitting a proportionality constant.

Beginning with Ravenstein in the late 1880s, his works proposed some
``Laws of Migration'' based on his empirical analysis of migration flows
in various countries
\citep{ravensteinLawsMigration1885, ravensteinLawsMigration1889}. These
works posited 1) a directly proportional relationship between migration
flows and the attractive size of destinations, and 2) an inversely
proportional relationship between the size of flows and the separation
between origins and destinations. As with Carey, these propositions echo
Newton's gravitational laws.

Over time, other researchers discovered similar relationships. For
example, Reilly \citep{reilly1929methods} formulated a law of retail
gravitation, expressed in terms of equal attraction to competing retail
destinations that could be understood as `potential trade territories'.
Later, Zipf proposed a \(\frac{P_1P_2}{D}\) hypothesis for the case of
information \citep{zipfDeterminantsCirculationInformation1946},
intercity personal movement \citep{zipfHypothesisIntercityMovement1946},
and goods movement by railways \citep{zipfHypothesisCaseRailway1946}.
The \(\frac{P_1P_2}{D}\) hypothesis stated that the magnitude of flows
was proportional to the product of the populations of settlements, and
inversely proportional to the distance between them.

Of the researchers cited above, only Reilly and Zipf expressed their
hypotheses in mathematical terms. Reilly's hypothesis was presented in
the following form:

\begin{equation}\phantomsection\label{eq-reilly}{
B_a = \frac{(P_a P_T)^N}{D_{aT}^n}
}\end{equation}

\noindent where \(B_a\) is the amount of business drawn to \(a\) from
\(T\), \(P_a\) and \(P_T\) are the populations of the two settlements,
and \(D_{aT}\) is the distance between them. Quantity \(N\) was chosen
by Reilly in a somewhat \emph{ad hoc} fashion as 1, and he used
empirical observations of shoppers to choose a value of \(n = 2\).

Zipf, on the other hand, wrote his hypothesis in mathematical form as:

\begin{equation}\phantomsection\label{eq-zipf}{
C^2 = \frac{P_1 P_2}{D_{12}}
}\end{equation}

\noindent where \(C\) is the volume of goods exchanged between \(1\) and
\(2\), \(P_1\) and \(P_2\) are the populations of the two settlements,
and \(D_{12}\) is the distance between them.

These early formulations (Eq~\ref{eq-reilly}, Eq~\ref{eq-zipf}) clearly
reflect the influence of Newtonian gravity on human spatial interaction
theory, revealing a shared mathematical structure across migration,
trade, and communication models.

However, a common feature of these early investigations is that none of
them included a proportionality constant (similar to \(G\) in
Eq~\ref{eq-phys-grav}), a consistent omission of the empirical
calibration necessary to convert these proportional relationships into
measurable and comparable quantities. It is only in Stewart (1948)
\citep{stewartDemographicGravitationEvidence1948} that we find the most
explicit connection yet to Newton's Gravitational law and the use of a
proportionality constant. While acknowledging predecessors like Reilly
and Zipf, the physicist Stewart was likely the first author to formalize
human spatial interaction using an explicit proportionality constant
\(G\), enabling his formulation to be interpreted as a measurable
`demographic' force:

\begin{equation}\phantomsection\label{eq-stewart-force}{
F = G\frac{(\mu_1N_1)(\mu_2N_2)}{d_{12}^2} = G\frac{M_1 M_2}{d_{12}^2} 
}\end{equation}

\noindent Where:

\begin{itemize}
\tightlist
\item
  \(F\) is the \emph{demographic force}
\item
  \(N_1\) and \(N_2\) are the numbers of people of in groups 1 and 2
\item
  \(\mu_1\) and \(\mu_2\) are so-called \emph{molecular weights}, the
  attractive weight of groups 1 and 2
\item
  \(M_1 = \mu_1N_1\) and \(M_2 = \mu_2N_2\) are the demographic masses
  at 1 and 2
\item
  \(d_{12}^2\) is the distance between \(1\) and \(2\)
\item
  And finally proportionality constant \(G\)
\end{itemize}

What is notable about Eq~\ref{eq-stewart-force}, however, is that the
proportionality constant \(G\) was specified but ``left for future
determination''
\citep[p.~34]{stewartDemographicGravitationEvidence1948}. We can infer
that it is crucial for ensuring \(F\) is maintained in some units of
demographic force.

In addition to demographic force \(F\), Stewart defined a measure of the
``potential'' of group \(2\) with respect to group \(1\). The partial
sum of the demographic force experienced by group \(1\), or the
potential number of people from location \(2\) that could visit location
\(1\), as \(V_1 = G\frac{M_2}{d_{12}}\). For a system with more than two
population bodies, Stewart formulated the population potential at \(i\)
by summing the contributions from each group \(j\), after arbitrarily
setting \(G=1\):

\begin{equation}\phantomsection\label{eq-stewart-population-potential-sum}{
V_i = \sum_j M_jd_{ij}^{-1}
}\end{equation}

\noindent Where \(M_j\) is the demographic mass at location \(j\) and
\(d_{ij}\) is the distance between \(i\) and \(j\). A version of this
discrete form is what was used in Hansen \citep{hansen1959}, going on to
become a foundation of modern accessibility definitions, as will be
discussed.

Although Stewart's concept of ``social physics'' eventually fell out of
favour, potentially in part due to inconsistent mathematical notation
(e.g., \(G\) is used as both a proportionality constant p.~34 and then
later as \emph{demographic energy} on p.~53.) as well as its racist and
unscientific assumptions (e.g., view of humans as particles following
physical laws and assumptions of the molecular weight of the average
American being one, but ``presumably\ldots much less than one\ldots.for
an Australian aborigine'' {[}p.~35{]}). However, Stewart's introduction
of a proportionality constant \(G\) in modeling demographic force marks
an important conceptual step: recognizing that moving from
proportionality to equality requires empirical calibration. In other
words, the addition of \(G\) shifts results from being abstract
indicators of potential (i.e.,
\(\frac{\text{people}^2}{\text{distance}^{2}}\)) to having units
grounded in consistent, albeit still abstract, quantities (i.e., units
of some sort of demographic force).

As will be discussed, when Hansen (1959) \citep{hansen1959} later
adopted Stewart's formula (Eq~\ref{eq-stewart-population-potential-sum})
for accessibility, he omitted any mention of \(G\), effectively setting
it to 1 arbitrarily as Stewart had done. This omission has persisted in
accessibility research, leaving a conceptual gap in how such measures
are interpreted and compared.

\section{4. Hansen's gravity-based accessibility to
today}\label{grav-to-today}

From Stewart \citep{stewartDemographicGravitationEvidence1948}, we
arrive at 1959 and Walter G. Hansen, whose work proved to be
exceptionally influential in the accessibility literature
\citep{hansen1959}. In this seminal paper, Hansen defined accessibility
as the potential of opportunities for interaction\ldots{} a
generalization of the population-over-distance relationship or
\emph{population potential} concept developed by Stewart
\citep{stewartDemographicGravitationEvidence1948}'' (p.~73). As well as
being a student of city and regional planning at the Massachusetts
Institute of Technology, Hansen was also an engineer with the Bureau of
Roads, and preoccupied with the power of transportation to shape land
uses in a very practical sense. Hansen \citep{hansen1959} drew directly
from Stewart's population potential formula (see
Eq~\ref{eq-stewart-population-potential-sum}), but left aside the
broader (and often problematic) framework of ``social physics''.

Hansen recast Stewart's population potential to reflect accessibility, a
model of human behaviour useful to capture regularities in mobility
patterns. Hansen replaced \(M_j\) in
Eq~\ref{eq-stewart-population-potential-sum} with \emph{opportunities}
to derive an \emph{opportunity potential}, or more specifically, a
\emph{potential of opportunities for interaction} as
\(S_{i} = \sum_j \frac{O_j }{d_{ij}^\beta}\). A contemporary rewriting
of \(d^{-\beta}\) to \(f(d_{ij})\) accounts for the variety of impedance
functions beyond the inverse power used in the applied literature:

\begin{equation}\phantomsection\label{eq-accessibility-general}{
S_{i} = \sum_j O_j  f(d_{ij})
}\end{equation}

\(S_{i}\) in Eq~\ref{eq-accessibility-general} is a measure of the
accessibility from zone \(i\). This is a function of \(O_j\) (the mass
of opportunities at \(j\)), \(d_{ij}\) (the cost, e.g., distance or
travel time, incurred to reach \(j\) from \(i\)), and \(f(d_{ij})\) (a
function that modulates the friction of cost). Today, Hansen is
frequently cited as the father of modern accessibility analysis
\citep[e.g.,][]{reggianiGuestEditorialNew2011}, and Hansen-type
accessibility is commonly referred to as the gravity-based accessibility
measure.

However, Hansen's formulation carried forward a crucial omission that
continues to affect the literature: the proportionality constant \(G\)
included in Stewart's original formulation
(Eq~\ref{eq-stewart-population-potential-sum}) has vanished entirely.
Although Stewart included \(G\) explicitly (with a note that ``\(G\)
{[}was{]} left for future determination: a suitable choice of other
units can reduce it to unity'' {[}p.~34{]}). Hansen made no mention of
it. As a result, modern accessibility analysis has largely evolved
without addressing the constant's role, leaving \(G\) effectively and
implicitly fixed at 1. This omission has significant implications.
Without a proportionality constant, the accessibility formula expresses
only a proportional relationship:
\(S_i \propto \sum_j g(O_j)f(d_{ij})\), not one of calibrated equality.
Recognition of the nature of this relationship is not common in the
literature, but is known, i.e., this proportional equation is shown in
Figure 1 in Wu and Levinson \citep{wuUnifyingAccess2020}.

Furthermore, working with a proportional relationship generates
fundamental issues in comparability between and, arguably, within
studies. Namely, accessibility estimates have no fixed unit, rendering
them sensitive to the choice of impedance functions. For instance, if
travel cost \(d_{ij}\) is measured in meters, then when the travel
impedance function \(f(d_{ij})\) equals \(d_{ij}^{-\beta}\), the
resulting \(S_i\) has units of opportunities per
\(\text{metres}^{\beta}\). However, when \(f(d_{ij})\) is set to equal
\(e^{-\beta d_{ij}}\), the units become opportunities per
\(e^{\beta \text{metres}}\). Such variation impairs comparability across
analyses and obscures the meaning of accessibility scores, making them
difficult to understand and communicate without post-hoc treatment.

Therefore, in practice, Hansen-type measures are ones of proportionality
and are better understood as \emph{ordinal indicators}; they rank
accessibility but lack cardinal meaning or consistent units
\citep{millerAccessibilityMeasurementApplication2018}. The continued
absence of a sort of proportionality constant \(G\) leaves a conceptual
and practical gap in accessibility analysis: a missing link between
theoretical form and empirical measurement that hinders the
interpretability of accessibility measures.

\section{5. Wilson's family of spatial interaction
models}\label{wilsons-family-of-spatial-interaction-models}

While accessibility research evolved in North America with Hansen
\citep{hansen1959}, a parallel development was taking place across the
Atlantic with Alan G. Wilson. Wilson's groundbreaking paper
\citep{wilson1971} advanced a general framework for spatial interaction
modeling focused on flows of interaction between places and derived from
a non-Newtonian analogy. Additionally, this work was not focused on the
`potential' concept as accessibility had been. Wilson \citep{wilson1971}
formalized the general spatial interaction model through the following
equation:

\begin{equation}\phantomsection\label{eq-phys-gravity-model}{
T_{ij} = k W_i^{(1)} W_j^{(2)} f(c_{ij})
}\end{equation}

The model in Eq~\ref{eq-phys-gravity-model} posits a quantity \(T_{ij}\)
that represents a value in a matrix of flows of size \(n \times m\),
that is, between \(i = 1,s, n\) origins and \(j = 1,s, m\) destinations.
The quantities \(W_i^{(1)}\) and \(W_j^{(2)}\) are proxies for the
masses at \(i=1,s,n\) origins and \(j=1,s,m\) destinations. The
super-indices \((1)\) and \((2)\) indicate that these masses can be any
number of different things associated with the zones, i.e.,
\(W_i^{(1)}\) could be population at a zone as an origin, and
\(W_j^{(2)}\) hectares of park space at a zone as a destination.
\(f(c_{ij})\) is some function of travel cost \(c_{ij}\) which reflects
travel impedance.

Of important note for this paper, \(k\) in
Eq~\ref{eq-phys-gravity-model} acts as a proportionality constant,
shifting the equation from a proportional to an equal relationship by
incorporating known system totals. In some sense, \(k\) serves a role
similar to the gravitational constant in Newton's law -- it calibrates
the model so that outputs match real-world quantities. However, these
real-world quantities are not a set empirical constant (like Newton's
\(G\)) but are instead sensitive to the system and known information
about the system.

From the outset, spatial interaction models emphasized interpretability
of results
\citep{kirbyNormalizingFactorsGravity1970, wilsonSTATISTICALTHEORYSPATIAL1967, wilson1971}.
But unlike earlier approaches that borrowed heuristically from Newtonian
gravity (i.e., interaction between masses over distance), Wilson's
innovation was to generalize the model using \emph{entropy
maximization}. By maximizing the number of ways individual trip
probabilities could be arranged under known constraints, Wilson derived
models that estimate \emph{statistical averages} of flows between zones
\citep{wilson1971, seniorGravityModellingEntropy1979}.

Crucially, to ensure that \(T_{ij}\) in Eq~\ref{eq-phys-gravity-model}
is maintained in units of flow between \(i\) and \(j\), the model moves
from proportionality to calibrated equality by incorporating empirical
constraints. At a minimum, this requires knowledge of the total number
of flows \(T\) in the system, leading to the basic constraint:

\begin{equation}\phantomsection\label{eq-constraint0-gravitymodel}{
\sum_i\sum_j T_{ij} = T
}\end{equation}

Additional information can be introduced. For example, when information
is available about the total number of flows produced by each origin,
\(W_i^{(1)}\) in Eq~\ref{eq-phys-gravity-model}, represented as \(O_i\),
then the following constraint can be used:

\begin{equation}\phantomsection\label{eq-constraint1-gravitymodel}{
\sum_j T_{ij} = O_i
}\end{equation}

As well, if there is information available about the total number of
flows attracted by each destination, \(W_j^{(2)}\) is represented as
\(D_j\) and the following constraint can be used:

\begin{equation}\phantomsection\label{eq-constraint2-gravitymodel}{
\sum_i T_{ij} = D_j
}\end{equation}

It is also possible to have information about both \(O_i\) and \(D_j\),
in which case both constraints can be imposed on the model at once.

Depending on which of the three system constraints are applied, a family
of spatial interaction models can be derived from
Eq~\ref{eq-phys-gravity-model}. The proportionality constant \(k\) is
replaced with different \emph{balancing factors}. This change in name is
more useful as these factors not only preserve proportionality but also
ensure that the predicted flows \(T_{ij}\) align with the known
constraints considered in the system. In other words, the different
balancing factors adjust the model so that flows become statistical
averages consistent with observed origin and/or destination data.

In the framework introduced and inferred from Wilson \citep{wilson1971},
three types of balancing factors are specified: (1) an unconstrained
model that only matches the total volume of interaction \(K\), (2) a
singly-constrained model (either by origins \(A_i\) or destinations
\(B_j\) -- explained below), and (3) a doubly-constrained model that
satisfies both.

In the unconstrained model, constraints in
Eq~\ref{eq-constraint1-gravitymodel} and
Eq~\ref{eq-constraint2-gravitymodel} do not hold. In practical terms,
this means that the total number of flows predicted by the model must be
equal to the sum of all flows from origins \(i\) to destinations \(j\).
The balancing factor \(K\) takes the place of \(k\) and is equal to the
following (as specified in \citep{cliff_evaluating_1974} and
\citep{fotheringham_spatial_1984}):

\begin{equation}\phantomsection\label{eq-total-flow-balancing-factor}{
K=\frac{T}{\sum_i\sum_j T_{ij}}
}\end{equation}

In the singly-constrained model, only constraint
Eq~\ref{eq-constraint1-gravitymodel} or constraint
Eq~\ref{eq-constraint2-gravitymodel} hold. When only
Eq~\ref{eq-constraint1-gravitymodel} holds, entropy maximization leads
to the production-constrained singly-constrained version of
Eq~\ref{eq-phys-gravity-model}, where the proxy for the mass at the
origin \(W_i^{(1)}\) is replaced with \(O_i\). Also, \(k\) is replaced
with a set of balancing factors specific to origins \(A_i\), which
ensures that constraint Eq~\ref{eq-constraint1-gravitymodel} is
satisfied (i.e., the sum of predicted flows from one origin going to all
destinations must equal the known mass at that origin \(O_i\)).
Satisfying this constraint also implicitly fulfills the total constraint
(Eq~\ref{eq-constraint0-gravitymodel}), since the sum of \(O_i\) values
across all origins equals the total number of flows. \(A_i\) takes the
following form:

\begin{equation}\phantomsection\label{eq-production-constrained-balancing-factor}{
A_i = \frac{1}{\sum_j W_j^{(2)} f(c_{ij})}
}\end{equation}

The singly-constrained attraction-constrained model is similar to the
production-constrained version but from the perspective of the mass at
the destination. For the attraction-constrained model, the proxy for the
mass at the destination \(W_j^{(2)}\) is replaced with \(D_j\) in
Eq~\ref{eq-phys-gravity-model}, representing the spatial interaction
inbound flow. Also, \(k\) is replaced with a set of destination-specific
balancing factors \(B_j\) that ensure that
Eq~\ref{eq-constraint2-gravitymodel} is satisfied (hence the total
constraint Eq~\ref{eq-constraint0-gravitymodel} is as well), meaning
that the sum of predicted flows going to one destination from all
origins must equal the known mass of that destination \(D_j\). As
before, destination-specific balancing factors \(B_j\) were derived by
Wilson as:

\begin{equation}\phantomsection\label{eq-attraction-constrained-balancing-factor}{
B_j = \frac{1}{\sum_i W_i^{(1)} f(c_{ij})}
}\end{equation}

Lastly, the doubly-constrained model is the production-attraction
constrained model in Wilson \citep{wilson1971}. In this case, both
constraints Eq~\ref{eq-constraint1-gravitymodel} and
Eq~\ref{eq-constraint2-gravitymodel} hold simultaneously. These
constraints ensure that the sum of predicted flows from one origin to
all destinations, and the predicted flows going to one destination from
all origins, must equal the known mass of the origin \(O_i\) and of the
destination \(D_j\). This should hold for all origins and destinations.
The resulting model is, in Wilson's terms, doubly-constrained, and from
Eq~\ref{eq-phys-gravity-model}, \(k\) becomes both \(A_i\) and \(B_j\)
shown in Eq~\ref{eq-doubly-constrained-balancing-factors}, and
\(W_i^{(1)}\) and \(W_j^{(2)}\) is replaced with \(O_i\) and \(D_j\).
Derivation of these models is demonstrated in detail elsewhere
\citep[e.g.,][]{ortuzar_2011_modelling, wilsonSTATISTICALTHEORYSPATIAL1967}.

\begin{equation}\phantomsection\label{eq-doubly-constrained-balancing-factors}{
\begin{array}{l}
A_i = \frac{1}{\sum_j B_j D_j f(c_{ij})}\\
B_j = \frac{1}{\sum_i A_i O_i f(c_{ij})}
\end{array}
}\end{equation}

Wilson's work is notable for many reasons. Rather than relying on a
universal constant like \(G\) or a scaling factor to balance units,
Wilson's models calibrate interaction flows through known empirical
constraints using principles of entropy maximization. This results in
interpretable, balanced (given the system knowns are expressed in the
constraints), and unit-consistent \(ij\) flows. The balancing factors
themselves have been subject to various interpretations -- as terminal
costs \citep{dieter1962distribution}, weighted mean values
\citep{kirbyNormalizingFactorsGravity1970}, or as accessibility measures
themselves \citep{cesario1977new} (as suggested in Wilson
\citep{wilson1971}) and as rents
\citep{morphetGravityModelCalibration2023} -- reflecting ongoing efforts
to understand what these mathematical constructs represent
behaviourally. In this way, the spatial interaction modeling tradition
can be seen to have succeeded where accessibility modeling stalled.
While Wilson's model produces results that are in units of flow tethered
to the system of analysis, which facilitates such interpretations,
Hansen-type measures (still widely used today in accessibility work)
yield outputs that reflect proportional -- not equal -- relationships
and typically lack interpretable units.

Before demonstrating the derivation of the family of accessibility
measures that are sensitive to constraints (Section 7), in the next
section, we review how conceptually intertwined the spatial interaction
modelling and accessibility literatures are, and where they began to
diverge. This investigation sheds light on why accessibility research
may have failed to adopt a comparable approach until this paper.

\section{6. Accessibility and spatial interaction modelling: two
divergent research
streams}\label{accessibility-and-spatial-interaction-modelling-two-divergent-research-streams}

Despite their close conceptual ties, the accessibility and spatial
interaction modeling literatures have developed along largely separate
paths since the 1970s. Hansen's \citep{hansen1959} formulation of
accessibility became the method used for decades of work on transport
equity, land use analysis, and urban accessibility planning. Meanwhile,
Wilson's \citep{wilson1971} entropy-maximizing framework reshaped how
spatial interaction models were constructed, particularly in
transportation demand forecasting. We argue the framework's quiet
innovation -- introducing empirically grounded constraints to shift from
proportionality to calibrated equality -- made the framework immediately
relevant for policy applications as outputs were in tangible units. Yet,
this mechanism was never widely adopted in accessibility analysis.

This divergence is especially striking given the context in which both
frameworks emerged. As noted in
\citep{battyChronicleScientificPlanning1994}, large-scale spatial
interaction models (like Wilson's) responded to important developments
at the time, a need ``to meet the dictates and needs of public policy
for strategic land use and transportation planning''. And these needs
were far from trivial: for instance, in the U.S., the Federal-Aid
Highway Act of 1956 set in motion the construction of the Interstate
Highway System with an eventual budget exceeding \$600 billion in
today's dollars
\citep{weinerUrbanTransportationPlanning2016, mdotMnDOTJoins2007}. In
this context, spatial interaction models were incorporated into
institutional practices focused on ``predict and provide'' travel demand
forecasting
\citep{kovatch1971modeling, weinerUrbanTransportationPlanning2016}.
Accessibility analysis, by contrast, remained more conceptually diffuse,
focused on indicators of ``potential'' spatial interaction with
opportunities rather than flows that could tangibly guide infrastructure
decisions (e.g., roadway capacity expansion, new construction). Whereas
spatial interaction modelling became a key element of transportation
planning practice, accessibility remained a somewhat more academic
pursuit, and the two streams of literature only rarely connected.

To explore why Wilson's approach may not have crossed over to
accessibility modeling sooner, we conducted a review of the literature
citing Hansen \citep{hansen1959}, Wilson \citep{wilson1971}, or both.
This was done using Web of Science's ``Cited References'' functionality,
and the digital object identifiers of Hansen \citep{hansen1959} and
Wilson \citep{wilson1971}. Only 76 out of the 2,122 documents that
emerged from our search of the General Database cite both. The number of
articles, by year and if they cite Hasen, Wilson, or both, are shown in
Figure 1.

Through the close analysis of why articles that cite both works cite
each work, we identify two distinct patterns: one group of articles
focused on accessibility, the other on spatial interaction. In examining
these groups of articles, we uncover how the relationship between Wilson
and Hansen has often been misunderstood, under-explored, or entirely
overlooked.

In the first stream of literature -- which cite both but are focused
more on spatial interaction models -- they treat spatial interaction and
accessibility as separate but related phenomena. Four subsets of this
stream emerge.

First, some of the more early works interpret the spatial interaction
model's balancing factors
(Eq~\ref{eq-production-constrained-balancing-factor} or
Eq~\ref{eq-doubly-constrained-balancing-factors}) as the inverse of
Hansen's accessibility measure
\citep{harrisEquilibriumValuesDynamics1978, leonardiOptimumFacilityLocation1978, fotheringhamSPATIALSTRUCTUREDISTANCE1981, fotheringhamSpatialCompetitionAgglomeration1985},
likely following Wilson's own recognition of this similarity between
balancing factor \(A_i\) and Hansen-type measure \(S_i\) on p.~10 in
Wilson \citep{wilson1971}. In some ways, this relationship has been
recognized as a ``common sense'' approach to incorporating accessibility
in the spatial interaction model
\citep[p.~99]{morrisAccessibilityIndicatorsTransport1979}, though
acknowledgment of its further exploration has been recommended
\citep{battyMethodResiduesUrban1976}.

The second subset of articles within this stream uses both Hansen
\citep{hansen1959} and Wilson's \citep{wilson1971} framework in
conjunction. For instance, some articles argue that spatial interaction
models fail to explain certain spatial patterns on their own, for
instance, as in Fotheringham
\citep{fotheringhamSpatialCompetitionAgglomeration1985} who demonstrates
how the spatial interaction model may insufficiently explain spatial
patterns, and suggests that explicitly defining destinations'
accessibility (Hansen-type accessibility) as a variable within the model
may remedy the issue (e.g., the \emph{competition destination} model).
Other works take a more applied approach: such as in defining
location-allocation problems in operations research
\citep{leonardiOptimumFacilityLocation1978, beaumontLocationallocationProblemsPlane1981},
estimating trips (or some other spatial interaction flows) alongside
accessibility
\citep[e.g.,][]{clarke2002deriving, grengs2004measuring, turk2019socio},
or considering accessibility as a variable within spatial interaction
models, in line with Fotheringham's
\citep{fotheringhamSpatialCompetitionAgglomeration1985} demonstration
\citep[e.g.,][]{beckers2022incorporating}.

The third subset of the spatial-interaction focused literature departs
from Hansen's \citep{hansen1959} definition, aligning instead with
microeconomic or utility-based interpretations of potential spatial
interaction e.g.,
\citep{morrisAccessibilityIndicatorsTransport1979, leonardiRandomUtilityDemand1984}.
Though across works in this subset, they recognize Hansen-type
accessibility as an indicator of `potential' but as a separate but
related concept to spatial interaction.

Moving on to the group of accessibility-focused literature that cites
both works, we categorize their citation of Wilson \citep{wilson1971}
within three general groups. Overall, these works do not engage -- or
only superficially engage -- with Wilson \citep{wilson1971}.

Firstly, there is a group of articles within this stream that cite
Wilson \citep{wilson1971} exclusively as attribution for using
context-dependent travel cost functions. This trend is common: for
instance, it is done in the following papers:
\citep{handyMeasuringAccessibilityExploration1997, weibullNumericalMeasurementAccessibility1980, kwan1998space, shenLocationCharacteristicsInnercity1998, ashiru2003space, rau2012spatial, pan2013impacts, margaridacondecomelhoradoImpactMeasuringInternal2016, caschili2015accessibility, grengs2015nonwork, pan2020measuring, chia2020extending, roblot2021participation, sharifiasl2023incorporating, kharel2024examining}.
However, these works do not engage with spatial interaction beyond this
attribution.

Secondly, a subset of literature explicitly acknowledge the conceptual
link between spatial interaction and with Hansen-type accessibility --
but only superficially, not going beyond the recognition that they both
relate to spatial interaction e.g.,
\citep{grengs2010job, millerMeasuringSpaceTimeAccessibility1999, giuliano2010accessibility, grengs2010intermetropolitan, grengs2012equity, levine2012does, levinson2012positive, tong2015transportation, liu2015spatial, he2017measuring, wuUnifyingAccess2020, ng2022reflection, naqavi2023mobility, suel2024measuring}.
Indeed, while accessibility can be seen as the \emph{potential} for
spatial interaction -- and Wilson \citep{wilson1971} briefly touches on
this -- such mentions have not resulted in deeper analytical integration
of these concepts. Furthermore, some of this literature also
occasionally conflates or blurs the distinction entirely, for instance,
by co-citing Hansen and Wilson as being `gravity models'
\citep[e.g.,][]{liu2004accessibility, dai2017visualization, shen2019segregation, chia2020extending}.
This conflation reveals ongoing murkiness between the distinction of
spatial interaction and the \emph{potential for} spatial interaction in
the literature.

Thirdly, there is a group of accessibility-focused works that interprets
the measure used in Hansen \citep{hansen1959} as the singly- or
doubly-constrained spatial interaction model's inverse balancing factor
\citep[e.g.,][]{vickermanAccessibilityAttractionPotential1974}. This
group often cites the spatial interaction works that make this
connection (i.e., the first subset of the first stream of literature)
and is especially prominent in the investigation of competitive
accessibility topics e.g.,
\citep{karstEvaluationAccessibilityImpacts2003, geurs2006accessibility, willigers2007accessibility, el2011place, curtis2010planning, manaugh2012makes, chen2013regional, alonso2014labour, albacete2017measuring, sahebgharani2019computing, mayaud2019future, allenMeasureCompetitiveAccess2020, levinsonGeneralTheoryAccess2020, marwal2022literature, su2023untangling}.
Only the works of Soukhov et al.
\citep{soukhovIntroducingSpatialAvailability2023, soukhovMultimodalSpatialAvailability2024}
use Wilson's \citep{wilson1971} balancing factors as a method for
maintaining constraints on opportunities within the context of
competitive accessibility.

On that note, Soukhov et al., 2023 and 2024
\citep{soukhovIntroducingSpatialAvailability2023, soukhovMultimodalSpatialAvailability2024}
introduce the balancing factors as mechanisms to ensure that
opportunities at each destination are proportionally allocated to each
zone (based on the proportion of population seeking opportunities and
the relative travel impedance). This is to ensure that each zonal
accessibility value is the sum of this proportional allocation from each
destination, and that all zonal values ultimately sum to the total
number of opportunities in the region. However, these balancing factors
were deduced intuitively. These works did not explicitly state that the
mathematical formulations of the equations are effectively equivalent to
Wilson's singly-constrained model (derived from entropy maximization).
This equivalence is only discovered in hindsight, as will be
demonstrated in the following section. These two works also do not
discuss other constrained cases that will also be addressed in this
paper.

In sum, despite the interpretative advantages offered by the statistical
logic of Wilson \citep{wilson1971}'s framework, neither the spatial
interaction literature that cites Hansen \citep{hansen1959} nor the
accessibility literature citing Wilson \citep{wilson1971} has
meaningfully applied Wilson's constraint-based intuition to the concept
of accessibility. So, in the next section, we do so by demonstrating how
the application of Wilson's framework enables accessibility to move from
a proportional relationship to calibrated equality -- tying outputs to
tangible system knowns. This re-expression of accessibility using
constraints yields interpretable, unit-consistent measures of
opportunity. This approach takes the same path of entropy-maximization
as in Wilson \citep{wilson1971}, and does not rely on specifying some
universal constant \(G\) like initially suggested in Stewart
\citep{stewartDemographicGravitationEvidence1948} (recall,
Eq~\ref{eq-stewart-population-potential-sum} which Hansen
\citep{hansen1959} operationalized).

\section{7. A family of accessibility measures: from proportionality to
equality}\label{a-family-of-accessibility-measures-from-proportionality-to-equality}

Despite their close conceptual ties, accessibility has not meaningfully
absorbed the constraint-based logic of spatial interaction modeling.
This section introduces this conceptual connection by defining a family
of accessibility measures using Wilson's approach grounded in
statistical mechanics, shifting place-based accessibility (\(\grave{a}\)
la Hansen \citep{hansen1959}) from a relationship describing
proportionality to one of equality. This shift addresses the issue of
unit interpretability (associated with the proportional nature of
Hansen-type accessibility indicators previously outlined).

We propose a revised definition of accessibility considerate of the
constraint-based spatial interaction model: \emph{the potential for
spatial interaction with opportunities (or population)}. We can specify
\(k\) as a type of proportional allocation factor \(\kappa\), which
incorporates Wilson's balancing factor(s) to define the \emph{potential
for spatial interaction with opportunities} \(V_{ij}\) and the
\emph{potential for spatial interaction with population} \(M_{ji}\). In
effect, \(\kappa\) is unitless and proportionally allocates (based on
the constraints of the case) opportunities (for \(V_{ij}\)) and
population (for \(M_{ji}\)). The equations are generally expressed as
follows:

\begin{equation}\phantomsection\label{eq-access-01}{
\begin{array}{l}
V_{ij}^X = \kappa_{ij}^X W_X^{(2)}\\ 
M_{ji}^X = \hat \kappa_{ji}^X W_X^{(1)}
\end{array}
}\end{equation}

\noindent Where \(W_X^{(2)}\) is the mass of the destination (i.e.,
opportunities \(O_j\) or \(O\)) and \(W_X^{(1)}\) is the mass of the
origin (i.e., population \(D_i\) or \(D\)) for either the zone or full
region, depending on the case (hence represented by a stand-in \(X\)
subindex). Accessibility flows can also be summarised as a partial sum
of the potential at \(i\) and at \(j\) to express accessibility at the
origin zone or at the destination zone, respectively. This form is
common in accessibility research:

\begin{equation}\phantomsection\label{eq-accesssibility-01}{
\begin{array}{l}
V_{i}^X = \sum_j \kappa_{ij}^X W_X^{(2)}\\
M_{j}^X = \sum_i \hat \kappa_{ji}^X W_X^{(1)}
\end{array}
}\end{equation}

Figure 2 illustrates our analytical framework using a 3-zone system. The
most detailed values, \(X_{ij}\), represent the potential for spatial
interaction from origin \(i\) to destination \(j\). Here, \(X\) stands
for all cases and variants to be discussed (e.g., \(V_{ij}^0\),
\(M_{ji}^0\), \(V_{ij}^T\), \(M_{ji}^T\), \(V_{ij}^S\), \(M_{ji}^S\),
\(V_{ij}^D\), and \(M_{ji}^D\)). Single marginals show the origin, and
destination weights and the total marginal is a sum of these values.

The proportional allocation constant \(\kappa\) takes the form of a
balancing factor that varies depending on the constraints applied. Each
member of the accessibility measure family is defined by the constraints
used, and can be grouped into the following four cases:

\begin{enumerate}
\def\labelenumi{\arabic{enumi}.}
\tightlist
\item
  \textbf{Unconstrained Case (\(V_i^0\), \(M_j^0\))}
\end{enumerate}

\begin{itemize}
\tightlist
\item
  Equivalent to Hansen's \citep{hansen1959} and Reilly's
  \citep{reilly1929methods} original formulations, the status quo of
  accessibility modelling.
\item
  No balancing factors applied; units are in
  ``opportunities-by-impedance'' for \(V_i^0\) or
  ``population-by-impedance'' for \(M_j^0\).
\item
  No constraints are applied, so values reflect proportionality only and
  are not calibrated to known system totals.
\end{itemize}

\begin{enumerate}
\def\labelenumi{\arabic{enumi}.}
\setcounter{enumi}{1}
\tightlist
\item
  \textbf{Total-constrained Case (\(V_i^T\), \(M_j^T\))}
\end{enumerate}

\begin{itemize}
\tightlist
\item
  Applies a total proportional allocation factor (\(\kappa_{ij}^T\),
  \(\hat \kappa_{ji}^T\)) based only on the total marginal (green box in
  Figure 2) i.e., total number of opportunities or population in the
  system. This ensures the sum of all values in the system match the
  total marginal.
\item
  Units of \(V_i^T\): accessible opportunities from \(i\), a value that
  is total-constrained and linearly proportional to \(V_i^0\).
\item
  Units of \(M_j^T\): accessible population from \(j\), a value that is
  total-constrained and linearly proportional to \(M_j^0\).
\end{itemize}

\begin{enumerate}
\def\labelenumi{\arabic{enumi}.}
\setcounter{enumi}{2}
\tightlist
\item
  \textbf{Singly-constrained Case (\(V_i^S\), \(M_j^S\))}
\end{enumerate}

\begin{itemize}
\tightlist
\item
  Applies singly-constrained proportional allocation factors
  (\(\kappa_{ij}^S\), \(\hat \kappa_{ji}^S\)) based on Wilson's
  balancing factors (\(B_j\), \(A_i\)) to preserve either the
  destination-side or origin-side marginal totals (blue and red boxes in
  Figure 2) i.e., the number of opportunities or population at each
  zone. Reflects how the literature calculates competitive
  accessibility.
\item
  Units of \(V_i^S\): accessible opportunities from \(i\), a value that
  is the sum of opportunity supply flows allocated proportionally based
  on demand at \(i\). Mathematically equivalent in per-capita form to
  2SFCA \citep{luo2003}.
\item
  Units of \(M_j^S\): accessible population from \(j\), a value that is
  the sum of population demand flows allocated proportionally based on
  supply at \(j\).
\end{itemize}

\begin{enumerate}
\def\labelenumi{\arabic{enumi}.}
\setcounter{enumi}{3}
\tightlist
\item
  \textbf{Doubly-constrained Case (\(V_{ij}^D\), \(M_{ji}^D\))}
\end{enumerate}

\begin{itemize}
\tightlist
\item
  Constrained on both origin and destination sides using both \(A_i\)
  and \(B_j\) simultaneously, which can also be expressed as
  proportional allocation factors (\(\kappa_{ij}^D\),
  \(\hat \kappa_{ji}^D\)); equivalent in interpretation to Wilson's
  \citep{wilson1971} doubly-constrained spatial interaction model.
\item
  Simultaneous application ensures both the destination-side \emph{and}
  origin-side marginal totals are maintained (blue and red boxes in
  Figure 2).
\item
  Interpretable only as \(ij\) and \(ji\) flows, since aggregation at
  \(i\) and \(j\) simply reproduces known totals. Represents
  `interaction capacity' or `realized access' serving as predictions of
  real interaction flows.
\end{itemize}

As a summary, each member of the family of accessibility measures is
named, explained in plain language, alongside their balancing factor(s),
proportional allocation factor(s), and mathematical equation and value
interpretations in Table 1.

\blandscape

Table 1: Summary of constrained accessibility measure types and
interpretations

\tiny
\begin{longtable*}{|p{2.5cm}|p{2.5cm}|p{2.5cm}|p{3cm}|p{3cm}|}
\hline
\textbf{Name of Member and Variant} & \textbf{Constraint Explanation and Balancing Factor} & \textbf{Proportional Allocation Factor} & \textbf{Accessibility Measure Equation} & \textbf{Interpretation} \\
\hline
\endfirsthead

\hline
\textbf{Name of member and Variant} & \textbf{Constraint Explanation and Balancing Factor} & \textbf{Proportional Allocation Factor} & \textbf{Accessibility Measure Equation} & \textbf{Interpretation} \\
\hline
\endhead

Unconstrained Accessible Opportunities ($V_i^0$) and Unconstrained Accessibile Population ($M_j^0$)
& No constraints; marginals not equal to any regional or zonal knowns.
& None
& $V_i^0 = \sum_j D_j f(c_{ij})$;

$M_j^0 = \sum_i O_i f(c_{ij})$

& Values in various units depending on the impedance and destination-mass (e.g., opportunities $\times$ decay) for $V_i^0$ and impedance and origin-mass (e.g., population $\times$ decay) for $M_j^0$; no total or marginal constraint \\
\hline

Total-constrained Accessible Opportunities ($V_i^T$) and Total Constrained Accessibile Population ($M_j^T$)
& Balancing factors $K^T$ and $\hat{K}^T$ ensures the sum of $ij$ values equals the total marginal, where:

$K^T = \frac{D}{\sum_i V_i^0}$;

$\hat{K}^T = \frac{O}{\sum_j M_j^0}$

& Allocates the total marginal as opportunities $D$ based on $\kappa_{ij}^T = \frac{W^{(2)}_j f(c_{ij})}{\sum_i\sum_j W^{(2)}_jf(c_{ij})}$ and as population $O$ based on $\hat \kappa_{ji}^T = \frac{W^{(2)}_i  f(c_{ij})}{\sum_i\sum_j W^{(2)}_if(c_{ij})}$

& $V^T_i =  \sum_j \kappa_{ij}^T D$;

$M_j^T =  \sum_i \hat{\kappa}_{ji}^T O$


& Values reflect a share of the total opportunities in the region $D$ for $V^T_i$ or total population in the region $O$ for $M^T_j$. \\
\hline

Singly-constrained Accessible Opportunities ($V_i^S$) and Singly Constrained Accessible Population ($M_j^S$)
& Single balancing factor $B_j$ (for $V_i^S$) that ensures the destination-mass marginal is constrained, and $A_i$ (for $M_j^S$) ensures the origin-mass marginal is constrained:

$B_j = \frac{1}{\sum_i W_i^{(1)} f(c_{ij})}$;

$A_i = \frac{1}{\sum_j W_j^{(2)} f(c_{ij})}$

& Allocates the single opportunities marginal $D_j$ proportionally based on $\kappa^S_{ij} =  \frac{W_i^{(1)} f(c_{ij})}{\sum_i W_i^{(1)} f(c_{ij})}$ in the case of $V_i^S$ and the single population marginal $O_i$ proroportionally based on $\hat \kappa^S_{ji} = \frac{W_j^{(2)} f(c_{ij})}{\sum_i W_j^{(2)} f(c_{ij})}$ in the case of $M_j^S$.

& $V^S_i = \sum_j \kappa^S_{ij} D_j$;

$M_j^S =  \sum_i \hat \kappa^S_{ji} O_i$

& $V^S_i$ values reflect a share of the opportunities at each destination $D_j$ based on origin population 'demand' and impedance; $M^S_j$ values reflect a share of the population at each origin $O_i$ based on destination opportunities 'supply' and impedance.\\
\hline

Doubly-constrained Access ($V_{ij}^D$ or $M_{ij}^D$)
& Values reflect both single marginals simutaneously, maintained via $A_i$ and $B_j$.
& —
& $V_{ij}^D = A_i B_j O_i D_j f(c_{ij})$
& The spatial interactions between population and opportunities (i.e., access). \\
\hline

\end{longtable*}

\elandscape

\subsection{7.1. Toy example setup}\label{toy-example-setup}

Consider the simple 3-zone region in Figure 2, where each zone serves as
both origin (\(i\)) and destination (\(j\)). The system includes three
inputs: zonal population and opportunities, a zonal cost (travel time)
matrix, and three travel impedance functions representing different
travel behaviours.

First, Table~\ref{tbl-small-system-land-use} summarizes population (in
10,000s) and physicians per zone. For context, the
Provider-to-Population Ratio (PPR) is 24.5, comparable to Canada's 2022
PPR of 24.97 physicians per 10,000 \citep{whoMedicalDoctors102025}.
Second, Table~\ref{tbl-small-system-cost} shows travel times (minutes);
it can be discerned that Zones 1 and 3 are closer to each other than to
Zone 2. Zones 1 and 3 together have a population roughly equal to Zone 2
but offer more than twice the physician availability. We interpret Zone
1 as the Urban Edge, Zone 3 as part of the Urban Core, and Zone 2 as
Suburban.

\begin{table}

\caption{\label{tbl-small-system-land-use}Simple system with three zones
(ID 1, 2 and 3). Population is in 10,000 persons and opportunities in
number of physicians.}

\centering{

\fontsize{7.5pt}{9.0pt}\selectfont
\begin{tabular*}{\linewidth}{@{\extracolsep{\fill}}rcc}
\toprule
ID (i or j) & Population\textsuperscript{\textit{1}} & Opportunities\textsuperscript{\textit{2}} \\ 
\midrule\addlinespace[2.5pt]
1 & 4 & 160 \\ 
2 & 10 & 150 \\ 
3 & 6 & 180 \\ 
\bottomrule
\end{tabular*}
\begin{minipage}{\linewidth}
\textsuperscript{\textit{1}}Population is \emph{Wi\textsuperscript{(1)}} when used as a proxy for the mass at the origin, and \emph{Oi} when used as a constraint.\\
\textsuperscript{\textit{2}}Opportunities is \emph{Wj\textsuperscript{(2)}} when used as a proxy for the mass at the destination, and \emph{Dj} when used as a constraint.\\
\end{minipage}

}

\end{table}%

\begin{table}

\caption{\label{tbl-small-system-cost}Cost matrix for system with three
zones (travel time in minutes).}

\centering{

\fontsize{7.5pt}{9.0pt}\selectfont
\begin{tabular*}{\linewidth}{@{\extracolsep{\fill}}cccc}
\toprule
 & \multicolumn{3}{c}{Destination ID} \\ 
\cmidrule(lr){2-4}
Origin ID & 1 & 2 & 3 \\ 
\midrule\addlinespace[2.5pt]
1 & 10 & 30 & 15 \\ 
2 & 30 & 10 & 25 \\ 
3 & 15 & 25 & 10 \\ 
\bottomrule
\end{tabular*}

}

\end{table}%

And third, Eq~\ref{eq-travel-behaviour-scenarios} presents the assumed
travel impedance functions reflecting three different travel behaviours.
A helpful analogy may be tying travel behaviour to the used mode's
mobility potential, i.e., the most decaying travel behaviour
(\(f_1(c_{ij})\)) would assume all travel in the region being done by
foot, while calculating accessibility assuming the least decay
(\(f_3(c_{ij})\)) would assume unfettered automobility.

\begin{equation}\phantomsection\label{eq-travel-behaviour-scenarios}{
\begin{array}{l}
f_1(c_{ij}) = \frac{1}{c_{ij}^3}\\
f_2(c_{ij}) = \frac{1}{c_{ij}^2}\\
f_3(c_{ij}) = \frac{1}{c_{ij}^{0.1}}
\end{array}
}\end{equation}

Any set of concepts representing population, opportunities, and their
associated travel behaviour, whether representing the entire region
uniformly (as will be demonstrated) or representing specific subgroups,
can be substituted into our simple toy example. The purpose of the
following simple example is to demonstrate the calculation and
interpretation of the four accessibility measure variants. As well, the
family of measures proposed will be available in the
\href{https://ipeagit.github.io/accessibility/}{\{accessibility\} R
package} for convenient use in future studies.

\subsection{7.2. Unconstrained
accessibility}\label{unconstrained-accessibility}

In \(V^0_i\), no proportional allocation factor is defined; simply
\(f(c_{ij})\) is used to weight the number of opportunities at each
\(j\) and the weighted values for each \(j\) are summed for each \(i\),
yielding an expression identical to Hansen's accessibility \(S_i\)
\citep{hansen1959}, the current standard practice in accessibility
measurement:

\begin{equation}\phantomsection\label{eq-unconstrained-accessibility}{
V^0_i = \sum_j V^0_{ij} = \sum_j W^{(2)}_jf(c_{ij}) = S_i
}\end{equation}

However, \(\sum_i V^0_{i}\) generally does not equal the total
opportunities \(O\), so units here are `opportunities weighted by travel
impedance' and lack meaningful scaling or direct interpretability.
Comparing values across different decay functions or contexts (i.e.,
different numbers of zones) is therefore limited to ordinal statements
(more vs.~less), not intervals or ratios (i.e., the magnitude of
differences).

\begin{table}

\caption{\label{tbl-simple-example-unconstrained-accessibility}Simple
system: unconstrained accessibility.}

\centering{

\fontsize{6.0pt}{7.2pt}\selectfont
\begin{tabular*}{\linewidth}{@{\extracolsep{\fill}}l|ccc}
\toprule
 & \multicolumn{3}{c}{V\textsubscript{i}\textsuperscript{0}} \\ 
\cmidrule(lr){2-4}
 & f\textsubscript{1} (c\textsubscript{ij}) = 1/c\textsubscript{ij}\textsuperscript{3} & f\textsubscript{2} (c\textsubscript{ij}) = 1/c\textsubscript{ij}\textsuperscript{2} & f\textsubscript{3} (c\textsubscript{ij}) = 1/c\textsubscript{ij}\textsuperscript{0.1} \\ 
\cmidrule(lr){2-2} \cmidrule(lr){3-3} \cmidrule(lr){4-4}
Origin & units: \emph{physicians-minute\^{}-3} & units: \emph{physicians-minute\^{}-2} & units: \emph{physicians-minute\^{}-0.1} \\ 
\midrule\addlinespace[2.5pt]
1 & 0.219 & 2.567 & 371.143 \\ 
2 & 0.167 & 1.966 & 363.479 \\ 
3 & 0.237 & 2.751 & 373.738 \\ 
\midrule 
\midrule 
Sum & 0.6233422 & 7.283556 & 1108.361 \\ 
\bottomrule
\end{tabular*}

}

\end{table}%

For example, Table~\ref{tbl-simple-example-unconstrained-accessibility}
shows \(V^0_{i}\) under each decay function. Comparing across decay
types is meaningless in absolute terms. For instance, the difference in
zone 1 (edge of urban core)'s accessibility under \(f_3\) vs \(f_1\) is
370.92, but in what units? These two values are a product of different
impedance functions (\(\text{physicians-minute}^{-0.1}\) and
\(\text{physicians-minute}^{-3}\)), making the direct comparison
uninterpretable (and arguably incorrect). The fundamental
uninterpretability of what is an
\emph{opportunity-weighted-travel-impedance} unit remains.

As the different impedance functions represent different travel
behaviours, comparing the raw unconstrained accessibility values across
groups is meaningless beyond notions of higher or lower. While one could
attempt to adjust the units post-calculation (e.g., scaling, population
normalization) or select impedance functions to facilitate comparison
across scenarios (potentially at the expense of accurately reflecting
travel behaviour), such adjustments may introduce bias. The
unconstrained scores are best used for ranking within a single context.

The next sections introduce constraints to calibrate these measures for
better interpretability and comparability, applying each to this example
in turn.

\subsection{7.3. Total-constrained
accessibility}\label{total-constrained-accessibility}

The total-constrained accessibility case can be interpreted in a few
ways. In the one that connects to the status quo: the total balancing
factor proportionally adjusts unconstrained zonal accessibility values
\(V^0_i\) so their total sum of \(V^0_i\) matches a known system total
-- either total opportunities or total population. Another
interpretation is in reformulating the equation to use a proportional
allocation constant based on the total balancing factor. The
proportional allocation constant distributes opportunities (or
population) proportionally by travel impedance.

In both formulations, all zonal values become a proportion of a known
system total, be it the regional opportunities or regional population,
depending on the variant.

We define two variants for this case:

\begin{itemize}
\tightlist
\item
  \(V_i^T\): accessibility is constrained by the total number of
  opportunities (total-constrained accessible opportunity), which is
  interpreted as Hansen's accessibility with a constraining constant,
  and
\item
  \(M_j^T\): where \(i\) and \(j\) of the first variant are transposed,
  yielding a measure constrained by the total population and to be
  interpreted as constrained `market potential'.
\end{itemize}

\subsubsection{7.3.1. Total-constrained accessible opportunities:
Hansen's accessibility with a total
constraint}\label{total-constrained-accessible-opportunities-hansens-accessibility-with-a-total-constraint}

In the total-constrained case, accessibility is expressed as a share of
the total number of opportunities in region \(D\), allocated based on
travel impedance. The total-constrained accessibility from \(i\) to
\(j\) takes the form:
\begin{equation}\phantomsection\label{eq-total-constrained-access}{
V^T_{ij} = \kappa_{ij}^T D
}\end{equation}

This formulation satisfies the total constraint, analogous to the one in
the Wilson framework:
\begin{equation}\phantomsection\label{eq-total-constraint-D}{
\sum_i \sum_j V^T_{ij} =  D
}\end{equation}

Next, the proportional allocation factor \(\kappa_{ij}^T\) determines
the share of total opportunities assigned to each origin--destination
pair, based on the relative proportion of opportunities-weighted travel
impedance: \[
\kappa_{ij}^T = \frac{W^{(2)}_j f(c_{ij})}{\sum_i\sum_j W^{(2)}_jf(c_{ij})}
\] This renders \(V_i^T\) (equal to \(\sum_jV_{ij}^T\)) into units of
opportunities (e.g., physicians), and allows direct interpretation and
comparison of results between zones and scenarios.

Alternatively, this formulation can be rewritten to be expressed using a
total-constrained balancing factor \(K^T\), which scales Hansen's
unconstrained accessibility \(V_i^0\) to meet the total opportunity
constraint:
\begin{equation}\phantomsection\label{eq-total-constrained-accessibility}{
V^T_i = \sum_j V^T_{ij} = K^T \sum_j W^{(2)}_jf(c_{ij}) = K^T  V^0_i
}\end{equation}

Where the total-constrained balancing factor \(K^T\) is:
\begin{equation}\phantomsection\label{eq-total-opportunity-balancing-factor}{
K^T = \frac{D}{\sum_i\sum_j W^{(2)}_jf(c_{ij})}
}\end{equation}

This expression is consistent with Wilson's entropy-maximizing framework
and analogous to the total flow spatial interaction model (e.g.,
Equation 2.11 in \citep{cliff_evaluating_1974}).

In summary, \(\kappa_{ij}^T\) proportionally allocates the total number
of opportunities \(D\) to each origin--destination pair based on
relative opportunity-weighted travel impedance. These values can be
aggregated across destinations to obtain total-constrained accessibility
at each origin. Alternatively, the measure can be expressed using the
balancing factor \(K^T\), demonstrating that it is algebraically
proportional to unconstrained accessibility \(V_i^0\), but with
interpretable units (i.e., opportunities). This allows for meaningful
comparisons of differences across zones and travel behaviour scenarios.

Referring back to our simple numeric example, \(K^T\) for the highest
decay travel scenario \(f_1(c_{ij}) = 1/c_{ij}^3\) would then be:

\[
K^T = \frac{D}{\sum_{i}\sum_{j} W_j^{(2)} f(c_{ij})} = \frac{490}{0.6233} = 786.085
\]

\(K^T\) for other decay scenarios are calculated similarly in code.
Applying each \(K^T\) to the unconstrained values \(V_i^0\) yields
total-constrained accessibility values
(Table~\ref{tbl-simple-example-total-opportunity-accessibility}), all in
units of physicians.

\begin{table}

\caption{\label{tbl-simple-example-total-opportunity-accessibility}Simple
system: total-constrained accessible opportunities.}

\centering{

\fontsize{6.0pt}{7.2pt}\selectfont
\begin{tabular*}{\linewidth}{@{\extracolsep{\fill}}l|ccc}
\toprule
 & \multicolumn{3}{c}{V\textsubscript{i}\textsuperscript{T}} \\ 
\cmidrule(lr){2-4}
 & f\textsubscript{1} (c\textsubscript{ij}) = 1/c\textsubscript{ij}\textsuperscript{3} & f\textsubscript{2} (c\textsubscript{ij}) = 1/c\textsubscript{ij}\textsuperscript{2} & f\textsubscript{3} (c\textsubscript{ij}) = 1/c\textsubscript{ij}\textsuperscript{0.1} \\ 
\cmidrule(lr){2-2} \cmidrule(lr){3-3} \cmidrule(lr){4-4}
Origin & units: \emph{physicians} & units: \emph{physicians} & units: \emph{physicians} \\ 
\midrule\addlinespace[2.5pt]
1 & 172.065 & 172.672 & 164.080 \\ 
2 & 131.627 & 132.247 & 160.692 \\ 
3 & 186.308 & 185.081 & 165.228 \\ 
\midrule 
\midrule 
Sum & 490 & 490 & 490 \\ 
\bottomrule
\end{tabular*}

}

\end{table}%

Compared to the unconstrained case, values now sum to the known regional
total \(D\), allowing interpretation of absolute and relative
differences across zones and travel scenarios. For example, in the
highest decay case, Zone 1 (Urban Edge) captures an intermediate number
of physicians (172.065), like in the unconstrained accessibility case.
However, unlike in the unconstrained case, we can say that this value is
out of the 490 physicians in the region, which allows us to deduce that
zone 1 captures 1.307 and 0.924 times more than zone 2 and 3. Values for
the lesser decay (\(f_2(c_{ij})\)) and lowest decay (\(f_3(c_{ij})\))
scenarios are calculated separately, with decay scenario values also
summing to 490 physicians accessible in the region.

One can also directly compare values at a specific zone, across travel
impedance scenarios, due to the consistent units. As the decay scenario
decreases, all zones become more accessible to each other and the
differences between pairs diminish (i.e., in \(f_3(c_{ij})\) each zone
captures close to an average amount of physicians, a third of 490 or
\textasciitilde163). In terms of proportional magnitude, this can also
be observed in the unconstrained measure for this scenario. However, for
the total-constrained measure, this plateauing of results has meaning.
In fact, each zone is allocated an average of the total amount in the
region, as a result of the total-constrained proportional allocation
factor.

However, what's notable is how zones change between scenarios. For
instance, Zone 1's share only declines slightly -- these declines are
outpaced by Zone 2's relative gains. This shift reflects how
\(\kappa_{ij}^T\) redistributes opportunities in proportion to impedance
under different travel behaviours.

The total-constrained accessibility measure resolves the
interpretability issue of Hansen's accessibility (i.e., unconstrained
accessibility) by grounding values in a meaningful total, enabling
robust comparisons across zones and scenarios, but also by keeping
values proportional to \(V_i^0\), so interpretation is similar.

\subsubsection{7.3.2. Total-constrained accessible population: Reilly's
potential trade territories with a total
constraint}\label{total-constrained-accessible-population-reillys-potential-trade-territories-with-a-total-constraint}

This second total-constrained variant is a transpose of the
opportunity-constrained formulation, switching indices \(i\) and \(j\)
to yield a measure of market potential -- the number of people who can
spatially interact with a destination.

Though not outlined in the ``Unconstrained accessibility'' section, the
unconstrained form aligns with Reilly's ``potential trade territories''
\citep{reilly1929methods} and Harris' and Vickerman's formulations of
regional market potential
\citep{harris_market_1954, vickermanAccessibilityAttractionPotential1974}.
In its unconstrained form, market potential has also been more recently
used to estimate potentially accessible populations following
infrastructure investments
\citep[e.g.,][]{gutierrezLocationEconomicPotential2001, holl2007twenty, condecco2018road}.
Market potential can also be thought of as a form of \emph{passive
accessibility}, indicating the number of people that can reach each
destination.

However, like \(V_{ij}^0\), issues of unit interpretability arise in
unconstrained market potential \(M_j^0\). To address this, we introduce
the total-constrained accessible population measure \(M^T_{ji}\), which
allocates the total population \(O\) across all origin-destination pairs
proportionally:

\begin{equation}\phantomsection\label{eq-total-constrained-market}{
M^T_{ji} = \kappa_{ji}^T O
}\end{equation}

\noindent Subject to the total constraint::
\begin{equation}\phantomsection\label{eq-total-constraint-O}{
\sum_i \sum_j M^T_{ji} =  O
}\end{equation}

Next, \(\hat \kappa_{ij}\) is the total-constrained proportional
allocation factor, a dimensionless term which distributes population
based on impedance-weighted accessibility: \[
\hat \kappa_{ij}^T = \frac{W^{(1)}_i f(c_{ij})}{\sum_i\sum_j W^{(1)}_if(c_{ij})}
\]

This renders \(M_j^T\) (equal to \(\sum_jM_{ji}^T\)) into units of
population. Alternatively, this formulation can be rewritten to be
expressed using a total-constrained balancing factor \(\hat K^T\), which
scales unconstrained market potential \(M_j^0\) to meet the total
population constraint:
\begin{equation}\phantomsection\label{eq-total-constrained-market-potential}{
M^T_j = \sum_i M^T_{ji} = \hat K^T \sum_i W^{(1)}_if(c_{ij}) = \hat K^T  M^0_j
}\end{equation}

Where the total-constrained balancing factor \(\hat K^T\) is:
\begin{equation}\phantomsection\label{eq-total-population-balancing-factor}{
\hat K^T = \frac{D}{\sum_i\sum_j W^{(1)}_if(c_{ij})}
}\end{equation}

In summary, \(\hat \kappa_{ij}^T\) allocates the total number of
population \(O\) proportionally to each origin--destination pair based
on relative population-weighted travel impedance. As well, the measure
can be expressed using the balancing factor \(\hat K^T\), demonstrating
that it is algebraically proportional to unconstrained market potential,
but also yielding interpretable units that allow for meaningful
comparison.

Returning to the numerical example, the balancing factor \(\hat K^T\) is
solved for each travel behaviour scenario, and the market potential of
each zone \(M^T_j\) is expressed as units of population (e.g., the
number of people accessible from each origin at that destination) in
Table~\ref{tbl-simple-example-total-population-accessibility}.

\begin{table}

\caption{\label{tbl-simple-example-total-population-accessibility}Simple
system: Total-constrained accessible population.}

\centering{

\fontsize{6.0pt}{7.2pt}\selectfont
\begin{tabular*}{\linewidth}{@{\extracolsep{\fill}}l|ccc}
\toprule
 & \multicolumn{3}{c}{M\textsubscript{i}\textsuperscript{S}} \\ 
\cmidrule(lr){2-4}
 & f\textsubscript{1} (c\textsubscript{ij}) = 1/c\textsubscript{ij}\textsuperscript{3} & f\textsubscript{2} (c\textsubscript{ij}) = 1/c\textsubscript{ij}\textsuperscript{2} & f\textsubscript{3} (c\textsubscript{ij}) = 1/c\textsubscript{ij}\textsuperscript{0.1} \\ 
\cmidrule(lr){2-2} \cmidrule(lr){3-3} \cmidrule(lr){4-4}
Destination & units: \emph{population in 10,000s} & units: \emph{population in 10,000s} & units: \emph{population in 10,000s} \\ 
\midrule\addlinespace[2.5pt]
1 & 5.018 & 5.447 & 6.598 \\ 
2 & 8.596 & 7.986 & 6.717 \\ 
3 & 6.386 & 6.567 & 6.684 \\ 
\midrule 
\midrule 
Sum & 20 & 20 & 20 \\ 
\bottomrule
\end{tabular*}

}

\end{table}%

Readers may note the difference in trends in accessible population
(Table~\ref{tbl-simple-example-total-population-accessibility}) and
accessible physicians (i.e., the preceding subsection,
Table~\ref{tbl-simple-example-total-opportunity-accessibility}).

In Table~\ref{tbl-simple-example-total-opportunity-accessibility}, zones
1-3 represent destinations, and the accessibility values reflect the
number of accessible people from the vantage of physicians. Zone 1, in
its role as a destination, is no longer intermediately ranked relative
to other zones; it now attracts the fewest number of people across all
three travel behaviour scenarios. However, similar to the
total-constrained opportunity case, as travel decay reduces, the
availability of population begins to converge (though Zone 1 continues
as the lowest-ranked) for similar reasons. As decay reduces, the
population's travel impedance to all zones becomes more similar, making
the relative location of the zones less important and all people in the
region more equally accessible.

Like in the total-constrained accessible opportunities variant, the
total-constrained accessible population enables direct comparison of raw
values, supporting both ordinal and interval interpretations across
space and travel behaviour scenarios.

\subsection{7.4. Singly-constrained
accessibility}\label{singly-constrained-accessibility}

The singly-constrained accessibility case can also be expressed in two
variants, each defined by the direction in which a constraint is
applied:

\begin{itemize}
\tightlist
\item
  \(V_i^S\): accessibility constrained by opportunities at destinations
  (singly-constrained accessible opportunities), and
\item
  \(M_j^S\): its transpose, constrained by population at origins
  (singly-constrained accessible population, or market potential).
\end{itemize}

Similar to the total-constrained case, the singly-constrained measures
adjust unconstrained zonal accessibility values (\(V_i^0\) or \(M_j^0\))
using a balancing factor to satisfy the known system constraint.
However, unlike the total constraint (which enforces a regional sum),
the singly-constrained case applies a localized constraint at one end of
the interaction -- either origin or destination.

In the opportunities-constrained variant \(V_i^S\), the balancing factor
ensures that only a proportion of opportunities at each destination are
allocated to origins, based on their relative demand (population) and
travel impedance. This variant mirrors the concept of spatial
availability as discussed in Soukhov et al.
\citep{soukhovIntroducingSpatialAvailability2023}. In the
population-constrained variant \(M_j^S\), the logic is reversed:
population at each origin is allocated proportionally across
destinations, informed by the distribution of opportunities and
impedance.

In both cases, the singly-constrained formulation introduces zonal-level
competition, unlike the total-constrained case, which distributes a
fixed regional sum. Each zonal accessibility value becomes not only a
fraction of the regional total (opportunities or population), but also a
balanced sum of interactions, weighted by impedance and relative
competition. The result remains in interpretable units -- accessible
opportunities or accessible population -- but reflects these more
complex spatial dynamics.

\subsubsection{7.4.1. Singly-constrained accessible opportunities:
spatial
availability}\label{singly-constrained-accessible-opportunities-spatial-availability}

In this singly-constrained variant, accessibility is constrained at the
destination side: the sum of accessible opportunities allocated from
each destination must equal the known number of opportunities \(D_j\).
This is comparable to the single attraction-constraint
(Eq~\ref{eq-constraint2-gravitymodel}) from Wilson's framework:

\begin{equation}\phantomsection\label{eq-opportunity-constraint}{
\sum_i V^S_{ij} =  D_j
}\end{equation}

The underlying spatial interaction model is now the
attraction-constrained model, and our accessibility measure becomes:

\begin{equation}\phantomsection\label{eq-opportunity-constrained-accessibility}{
V^S_{i} = \sum_j B_j D_j W_i^{(1)} f(c_{ij})
}\end{equation}

\noindent where \(W_i^{(1)}\) is a measure of the mass at origin \(i\)
(i.e., the opportunity-seeking population). The corresponding balancing
factor, as per Wilson, is:

\begin{equation}\phantomsection\label{eq-opportunity-constrained-proportionality-constants}{
B_j = \frac{1}{\sum_i W_i^{(1)} f(c_{ij})}
}\end{equation}

Introducing the balancing factor in
Eq~\ref{eq-opportunity-constrained-accessibility}, we obtain:

\begin{equation}\phantomsection\label{eq-opportunity-constrained-accessibility-with-balancing-factor}{
V^S_{i} = \sum_j D_j \frac{W_i^{(1)} f(c_{ij})}{\sum_i W_i^{(1)} f(c_{ij})}
}\end{equation}

Further, we can express the formula even more simply by defining the
following proportional allocation factor:

\begin{equation}\phantomsection\label{eq-opportunity-constrained-proportional-allocation-factor}{
\kappa^S_{ij} = \frac{W_i^{(1)} f(c_{ij})}{\sum_i W_i^{(1)} f(c_{ij})}
}\end{equation}

After this, it is possible to rewrite
Eq~\ref{eq-opportunity-constrained-accessibility-with-balancing-factor}
as an origin summary expression of proportionally allocated known
opportunities (i.e., \(D_j\)).

\begin{equation}\phantomsection\label{eq-attraction-constrained-accessibility-with-proportional-allocation-factor}{
V^S_{i} = \sum_j \kappa^S_{ij} D_j
}\end{equation}

This formulation has been referred to as \textbf{spatial availability}
by Soukhov et al. \citep{soukhovIntroducingSpatialAvailability2023},
since it incorporates spatial competition by allocating opportunities
based on demand (population), impedance, and the known opportunity
totals \(D_j\). The dimensionless factor \(\kappa^S_{ij}\) ensures that
each destination's opportunities are distributed proportionally to
origins. As in the total-constrained case, \(V_i^S\) is expressed in the
units of accessible opportunities.

Soukhov et al. \citep{soukhovIntroducingSpatialAvailability2023} also
showed that the following expression (accessibility per capita) is a
constrained version of the popular 2SFCA approach of Shen
\citep{shen1998} and Luo and Wang \citep{luo2003}:

\begin{equation}\phantomsection\label{eq-opportunity-constrained-accessibility-per-capita}{
v^S_{i} = \frac{V^S_{i}}{W^{(1)}_i}
}\end{equation}

Returning to the simple numeric example, as an example of the solved
\(B_{j}\) for the highest decay travel behaviour \(f_1(c_{ij})\):

\[
\begin{array}{l}
B_{j} = \frac{1}{\sum_i W_i^{(1)} f(c_{ij})}\\
B_{1} =  \frac{1}{\frac{4}{10^3} + \frac{10}{30^3} + \frac{6}{15^3}} = 162.6506\\ 
B_{2} =  \frac{1}{\frac{4}{30^3} + \frac{10}{10^3} + \frac{6}{25^3}} = 94.9474\\
B_{3} =  \frac{1}{\frac{4}{10^3} + \frac{10}{25^3} + \frac{6}{10^3}} = 93.9850
\end{array}
\]

The balancing factors \(B_j\) for the \(f_2(c_{ij})\) decay group for
zones 1, 2 and 3 are 12.857, 8.769 and 10.664, respectively. For the
\(f_3(c_{ij})\) decay group, they are 0.067, 0.066 and 0.066. Using
these balancing constants, we can calculate the singly-constrained
opportunity accessibility
(Table~\ref{tbl-simple-example-attraction-constrained-accessibility}).

\begin{table}

\caption{\label{tbl-simple-example-attraction-constrained-accessibility}Simple
system: singly-constrained accessible opportunities.}

\centering{

\fontsize{6.0pt}{7.2pt}\selectfont
\begin{tabular*}{\linewidth}{@{\extracolsep{\fill}}l|cccc}
\toprule
 &  & \multicolumn{3}{c}{V\textsubscript{i}\textsuperscript{S}} \\ 
\cmidrule(lr){3-5}
 &  & f\textsubscript{1} (c\textsubscript{ij}) = 1/c\textsubscript{ij}\textsuperscript{3} & f\textsubscript{2} (c\textsubscript{ij}) = 1/c\textsubscript{ij}\textsuperscript{2} & f\textsubscript{3} (c\textsubscript{ij}) = 1/c\textsubscript{ij}\textsuperscript{0.1} \\ 
\cmidrule(lr){3-3} \cmidrule(lr){4-4} \cmidrule(lr){5-5}
Origin & Population (10k) & units: \emph{physicians} & units: \emph{physicians} & units: \emph{physicians} \\ 
\midrule\addlinespace[2.5pt]
1 & 4 & 133.469 & 122.255 & 98.848 \\ 
2 & 10 & 166.781 & 185.096 & 241.877 \\ 
3 & 6 & 189.750 & 182.650 & 149.275 \\ 
\midrule 
\midrule 
Sum & — & 490 & 490 & 490 \\ 
\bottomrule
\end{tabular*}

}

\end{table}%

Imposing the single proportional allocation factor \(\kappa^S_{ij}\)
allows for the comparison of differences and ratios of the accessibility
values, like previously discussed in the total-constrained accessible
opportunities case. The proportional allocation factor ensures that
resulting values are in units of \emph{physicians}, with the impedance
units already accounted for in the allocation process.

However, unlike \(\kappa^T_{ij}\), \(\kappa^S_{ij}\) introduces zonal
competition based on the mass of the origin (population). In the
total-constrained case, opportunities are distributed based on impedance
alone, regardless of population at \(i\). In contrast, the
singly-constrained case allocates each zone's opportunities
proportionally across the region based on the relative
impedance-weighted demand from all origins.

This consideration has important implications. Consider the highest
decay scenario \(f_1(c_{ij})\). Under this scenario, Zone 1 -- despite
hosting a medium amount of physicians -- captures the fewest physicians
(133.469), compared to 166.781 at Zone 2, and 189.75 at Zone 3. Why?
Zone 1 has the smallest population and is adjacent to Zone 3, the urban
core. Its low impedance-weighted demand means \(\kappa^S_{ij}\)
allocates it fewer opportunities. By contrast, in the total-constrained
case, Zone 1 fares better, capturing 35\% of all physicians (compared to
27\%).

As travel decay increases (e.g., from \(f_3(c_{ij})\) to
\(f_1(c_{ij})\)), competition becomes less diffuse. Zone 2, with the
largest population, claims the most opportunities under \(f_3(c_{ij})\).
But under lower decay, Zones 1 and 3 draw relatively more opportunities
from Zone 2. For instance, Zone 1 gains 6\% more from Zone 2 in
\(f_1(c_{ij})\) than in \(f_3(c_{ij})\). This shift reflects a drop in
\(\kappa^S_{2,2}\) of 14\%, reflecting Zone 2's decreasing hold on its
own opportunities as other zones gain accessibility `parity'.

This dynamic reveals how \(\kappa^S_{ij}\) embeds both travel impedance
and population competition. Unlike the total constraint lowest decay
scenarios that allocate evenly (as localized supply and demand are
assumed to be unknown), the singly-constrained case reflects
competition's influence on allocation.

In this way, the consideration of constrained accessibility \emph{per
capita} may be clarifying. Often, accessibility values are reported as
raw scores without considering the population potentially accessing
them. But, as we introduced constraints, these constrained accessibility
values can be normalized using anything that is relevant to the zone. In
Table~\ref{tbl-simple-example-attraction-constrained-accessibility-per-capita},
we present per capita accessibility for the numeric example simply in
units reflecting the number of physicians accessible per population at
each zone. Notably, these per capita rates are equivalent to the 2SFCA
values.

\begin{table}

\caption{\label{tbl-simple-example-attraction-constrained-accessibility-per-capita}Simple
system: Singly-constrained accessible opportunities per capita.}

\centering{

\fontsize{6.0pt}{7.2pt}\selectfont
\begin{tabular*}{\linewidth}{@{\extracolsep{\fill}}l|cccc}
\toprule
 &  & \multicolumn{3}{c}{v\textsubscript{i}\textsuperscript{S}} \\ 
\cmidrule(lr){3-5}
 &  & f\textsubscript{1} (c\textsubscript{ij}) = 1/c\textsubscript{ij}\textsuperscript{3} & f\textsubscript{2} (c\textsubscript{ij}) = 1/c\textsubscript{ij}\textsuperscript{2} & f\textsubscript{3} (c\textsubscript{ij}) = 1/c\textsubscript{ij}\textsuperscript{0.1} \\ 
\cmidrule(lr){3-3} \cmidrule(lr){4-4} \cmidrule(lr){5-5}
Origin & Population (10k) & units: \emph{physicians per capita} & units: \emph{physicians per capita} & units: \emph{physicians per capita} \\ 
\midrule\addlinespace[2.5pt]
1 & 4 & 33.367 & 30.564 & 24.712 \\ 
2 & 10 & 16.678 & 18.510 & 24.188 \\ 
3 & 6 & 31.625 & 30.442 & 24.879 \\ 
\bottomrule
\end{tabular*}

}

\end{table}%

This simple example was constructed so that the regional average equals
24.5 physicians per 10,000 people. As distance decay decreases and
becomes \emph{relatively} uniform (all zones can reach all zones), the
effect of population drives the proportional allocation of
opportunities. Consequently, per capita accessibility values begin to
stabilize to the regional per capita average (e.g., in the highest
distance decay \(f_1(c_{ij})\), per capita values are all
\textasciitilde24 physicians accessible).

This convergence mirrors the trend in the total-constrained opportunity
case, where accessibility values approach a third of the 490 physicians
under the lowest-decay unfettered mobility scenario \(f_3(c_{ij})\). In
both cases, the balancing factors (\(K^S\) and \(B_j\)) act as averaging
mechanisms but at different scales. As distance decay becomes
\emph{relatively} more uniform, the role of remaining variables (i.e.,
total population or opportunities) drive the proportional allocation
differences. In the total-constrained case, this is the proportion of
opportunities relative to the regional opportunities, and in the case of
the single opportunity constrained case, this is the population at a
zone relative to the regional population.

\subsubsection{7.4.2. Singly-constrained accessible population: market
availability}\label{singly-constrained-accessible-population-market-availability}

Similar to Eq~\ref{eq-total-population-balancing-factor} in transposing
the origins and destinations, we can define a \emph{singly-constrained}
measure of market potential that preserves the known population (i.e.,
the mass weight at the origin \(W_i^{(1)}\) is now represented by
\(O_i\)). In its per-capita expression, i.e., equivalent to 2SFCA, this
constrained concept of market potential has been used to express
``facility crowdedness'' as in Wang \citep{wang_inverted_2018}.

The underlying spatial interaction model is now the
production-constrained model version of Eq~\ref{eq-phys-gravity-model},
and our market potential measure \(M^S_j\) becomes:

\begin{equation}\phantomsection\label{eq-population-constrained-accessibility}{
M^S_j = \sum_i A_i O_i W_j^{(2)} f(c_{ij})
}\end{equation}

In this variant, the measure is singly-constrained by the population
\emph{by origin} (i.e., \(O_i\)), like
Eq~\ref{eq-constraint2-gravitymodel} from Wilson's framework:

\begin{equation}\phantomsection\label{eq-population-constraint}{
\sum_j M^S_{ji} =  O_i 
}\end{equation}

And the corresponding balancing factor, as per Wilson, is:

\begin{equation}\phantomsection\label{eq-population-constrained-proportionality-constants}{
A_i = \frac{1}{\sum_j W_j^{(2)} f(c_{ij})}
}\end{equation}

Following the same logic as in the preceding section on
total-constrained market potential, one arrives at the following
expression of accessible population \(M_j^S\) being the product of
proportionally allocated (\(\hat \kappa^S_{ji}\)) population:

\begin{equation}\phantomsection\label{eq-production-constrained-accessibility-with-proportional-allocation-factor}{
M^S_{j} = \sum_i \hat \kappa^S_{ji} O_i
}\end{equation}

\noindent with:

\begin{equation}\phantomsection\label{eq-attraction-constrained-proportional-allocation-factor}{
\hat \kappa^S_{ji} = \frac{W_j^{(2)} f(c_{ij})}{\sum_i W_j^{(2)} f(c_{ij})}
}\end{equation}

As well, the single (population) constraint in
Eq~\ref{eq-population-constraint} ensures that the total constraint
(e.g., \(\sum_j M^S_{j} = \sum_i\sum_j  M^S_{ji} = O\)) is maintained.

With these constraints, \(\frac{M_j^S}{O}\) can be interpreted as the
proportion of the total population serviced by location \(j\).

For the sake of brevity, we'll move on to the doubly-constrained case.

\subsection{7.5. Doubly-constrained
accessibility}\label{doubly-constrained-accessibility}

This accessibility case adopts the structure of the doubly-constrained
spatial interaction model, where \(V_{ij}^D\) flows are constrained by
both origin populations \(O_i\) and destination opportunities \(D_j\).
That is, the resulting accessibility outflow from each origin must match
the origin's population demand, and the resulting accessibility inflow
to each destination must match the number of opportunities supplied:

\begin{equation}\phantomsection\label{eq-opportunity-population-equality-1}{
\sum_j V_{ij}^D = O_i \text{ and }  \sum_i V_{ij}^D =  D_j
}\end{equation}

Because results are made to match values in both marginals, the results
cannot be interpreted as a traditional summary at \(i\) or \(j\) (e.g.,
``opportunities accessible from \(i\)''), as evidently those sums simply
reproduce themselves. Instead, the meaningful unit of analysis is the
\(ij\) flow itself.

This distinguishes doubly-constrained accessibility from the total and
singly-constrained cases discussed previously. In those cases, only one
side of the interaction -- either the total marginal or
opportunity/population marginal -- was constrained, while the other was
treated as a demand/supply weight (e.g., \(D\) or \(O\) for
total-constrained and \(W_j^{(2)}\) or \(W_i^{(1)}\) for
singly-constrained).

By contrast, the doubly-constrained model assumes both the demand
(population) and the supply (opportunity) are known and bounded, and
allocates flows accordingly. This makes it less suitable for traditional
accessibility analysis, namely because origin and destination masses
often differ in kind and units. For instance, the number of people
accessing an opportunity, such as a park, may be known, but the capacity
of each park is not. A doubly-constrained approach may make conceptual
sense if the ``potential'' should be contained only in the flows
themselves, meaning the units of population and opportunities are
comparable, have a one-to-one correspondence or are otherwise paired
(e.g., job per worker, student per school seat, or vaccine doses per
person). Mathematically, this model requires the simultaneous imposition
of both the population- and opportunity- constraints in the preceding
singly-constrained variants (Eq~\ref{eq-opportunity-constraint} and
Eq~\ref{eq-population-constraint}), namely the sum of population in all
origins should match the sum of opportunities in all destinations
(Eq~\ref{eq-opportunity-population-equality-2}):

\begin{equation}\phantomsection\label{eq-opportunity-population-equality-2}{
\sum_i O_i = \sum_j D_j
}\end{equation}

As before, the simultaneous imposition of both constraints ensures the
total system constraint is maintained i.e.,
\(\sum_i V^D_{i} = \sum_i\sum_j  V^D_{ij} = D\) remains equal to the
total number of opportunities in the region \(O\).

The doubly-constrained accessibility measure \(V_{ij}^D\) takes the form
of the production-attraction (doubly-constrained) spatial interaction
model as follows:

\begin{equation}\phantomsection\label{eq-doubly-constrained-accessibility}{
V_{ij}^D = A_i B_j O_i D_j f(c_{ij})
}\end{equation}

\noindent where the corresponding balancing factors \(A_i\) and \(B_j\),
as per Wilson, are:

\[
\begin{array}{l}
A_i = \frac{1}{\sum_j B_j D_j f(c_{ij})}\\
B_j = \frac{1}{\sum_i A_i O_i f(c_{ij})}
\end{array}
\]

Calibration of the two sets of proportionality constants is accomplished
by means of iterative proportional fitting, whereby the values of
\(A_i\) are initialized as 1 for all i to obtain an initial estimate of
\(B_j\). The values of \(B_j\) are used to update the underlying
\(V_{ij}^D\) matrix, before calibrating \(A_i\). This process continues
to update \(A_i\) and \(B_j\) until a convergence criterion is met
\citep[see][p.~193-195]{ortuzar_2011_modelling}.

The doubly-constrained model completely distributes origin populations
to destination opportunities according to travel impedance and
supply-demand balance. This ensures that: summing \(V^D_{ij}\) across
\(j\) returns \(O_i\); summing across \(i\) returns \(D_j\). Thus,
aggregating over \(i\) or \(j\) yields only the known constraints. In
this way, a per-capita form (e.g., \(V^D_i / O_i\)) is not
meaningful--since the output already reflects population-normalized
allocation. As such, the \(ij\) matrix \(V^D_{ij}\) is the only
interpretable output.

We could define the proportional allocation factor \(\kappa_{ij}^D\)
such that:

\[
\kappa_{ij}^D = \sum_j \frac{1}{\sum_j B_j D_j f(c_{ij})} \frac{1}{\sum_i A_i O_i f(c_{ij})} O_i f(c_{ij})
\] \noindent and represent \(V^D_{ij}\) as equal to
\(\kappa^D_{ij} D_j\), allowing the analyst to understand the
proportional allocation of \(D_j\)s to each \(ij\) flow.

Following this logic, the market potential from \(M^D_{ji}\) is
effectively equivalent to \(V_{ij}^D\), but can be read with a different
interpretation: the opportunities accessed from \(j\) at an \(i\)
vs.~the population accessed from \(i\) at a \(j\). The inputs of
`opportunities accessed' and `accessed population' can already be
interpreted as inherently being sensitive to both opportunities and
population.

To calculate doubly-constrained accessibility using the toy example, the
interpretation of the population data and the counts of the opportunity
data must be reworked. Namely, a count of physician \emph{capacity} per
destination \(D_j\) is needed instead of simply the number of
physicians. We also need to be able to clearly state that the population
is the \emph{capacity} of the origin to interact with opportunities
\(O_i\), i.e., the count of people seeking opportunities.

This adjustment to the example is summarised in
Table~\ref{tbl-small-system-land-use-doubly-constrained-case}. With the
population (in units of 10,000s of people seeking physicians) and the
opportunities (in units of 10,000s of physician-capacity) per zone. For
the population, we leave this unchanged numerically, but we now must
keep in mind that each person interacts with one unit of physician
capacity. The number of providers per destination is, however, revised
to represent physician capacity, scaled approximately from the original
number of physicians used in previous cases
(Table~\ref{tbl-small-system-land-use}). The system-wide PPR is now 1
(recall: the unmodified example's system PPR is 24.5).

We keep the same zonal cost matrix and travel impedance functions for
three types of travel behaviour as before
(Table~\ref{tbl-small-system-cost} and
Eq~\ref{eq-travel-behaviour-scenarios}).

\begin{table}

\caption{\label{tbl-small-system-land-use-doubly-constrained-case}Modified
simple system with three zones reflecting matched population and
opportunities. Population is in 10,000 persons and opportunities in
10,000 of physician-capacity.}

\centering{

\fontsize{7.5pt}{9.0pt}\selectfont
\begin{tabular*}{\linewidth}{@{\extracolsep{\fill}}rcc}
\toprule
ID (i or j) & Population & Opportunities \\ 
\midrule\addlinespace[2.5pt]
1 & 4 & 7 \\ 
2 & 10 & 5 \\ 
3 & 6 & 8 \\ 
\bottomrule
\end{tabular*}

}

\end{table}%

And with these modifications to the example, our objective is slightly
different: to predict the flows from \(j\) knowing that the amount of
physician-capacity at each \(j\) must be preserved and all flows to
\(i\) should match the number of people at \(i\), under different travel
behaviour scenarios. Put another way, we're interested in the \(ij\)
flows assuming we already know accessibility at each \(i\). The highest
decay travel behaviour scenario (\(f_1(c_ij)\)) is presented in
Table~\ref{tbl-adjusted-small-system-land-use-doubly-constrained-case-f1cij-access-values}.

\begin{table}

\caption{\label{tbl-adjusted-small-system-land-use-doubly-constrained-case-f1cij-access-values}Doubly-constrained
accessible opportunities assuming highest travel decay in the modified
simple system.}

\centering{

\fontsize{7.5pt}{9.0pt}\selectfont
\begin{tabular*}{\linewidth}{@{\extracolsep{\fill}}l|rcccc}
\toprule
 &  & \multicolumn{3}{c}{Destination ID} &  \\ 
\cmidrule(lr){3-5}
 & Origin ID & 1 & 2 & 3 & sum \\ 
\midrule\addlinespace[2.5pt]
 & 1 & 3.235859 & 0.01032226 & 0.7556568 & 4 \\ 
 & 2 & 2.132602 & 4.95932483 & 2.9044391 & 10 \\ 
 & 3 & 1.631539 & 0.03035291 & 4.3399040 & 6 \\ 
\midrule 
\midrule 
Sum & — & 7 & 5 & 8 & — \\ 
\bottomrule
\end{tabular*}

}

\end{table}%

As shown in
Table~\ref{tbl-adjusted-small-system-land-use-doubly-constrained-case-f1cij-access-values}
for the highest-decay scenario \(f_1(c_{ij})\), accessibility is no
longer meaningfully represented as zonal summaries like \(V^D_i\) or
\(M^D_j\), since these values reproduce the original constraints, i.e.,
\(V^D_i = O_i\), hence the physician-capacity accessible for Zones 1, 2,
and 3 are 4.002, 9.996, and 6.002. The usefulness of the
doubly-constrained measure lies in the interpretation as \(V_{ij}^D\)
values; the number of opportunities from zone \(j\) allocated to
populations in zone \(i\) is shaped by both mass and travel impedance.

To illustrate this concept, consider the results for Zone 2 (recall: a
suburban type zone, with a higher population, lower amount of
opportunities, and relatively remote). As shown in
Table~\ref{tbl-adjusted-small-system-land-use-doubly-constrained-case-allfs-access-values-for-zone2},
its intrazonal flow (i.e., \(V^D_{22}\)) declines as travel impedance
decay decreases -- from 4.959 under \(f_1(c_{ij})\) to 2.667 under
\(f_3(c_{ij})\), out of the \textasciitilde10 opportunities allocated to
Zone 2 (a population of 10).

Following the intuition discussed in the singly-constrained opportunity
case, as decay decreases, the mass effects (effect of the population and
opportunities magnitudes) become relatively more dominant in the spatial
allocation.

\begin{table}

\caption{\label{tbl-adjusted-small-system-land-use-doubly-constrained-case-allfs-access-values-for-zone2}Doubly-constrained
accessible opportunities at Zone 2 for all travel decay groups in the
modified simple system.}

\centering{

\fontsize{7.5pt}{9.0pt}\selectfont
\begin{tabular*}{\linewidth}{@{\extracolsep{\fill}}>{\raggedright\arraybackslash}p{\dimexpr 22.50pt -2\tabcolsep-1.5\arrayrulewidth}|>{\centering\arraybackslash}p{\dimexpr 67.50pt -2\tabcolsep-1.5\arrayrulewidth}>{\centering\arraybackslash}p{\dimexpr 67.50pt -2\tabcolsep-1.5\arrayrulewidth}>{\centering\arraybackslash}p{\dimexpr 67.50pt -2\tabcolsep-1.5\arrayrulewidth}>{\centering\arraybackslash}p{\dimexpr 67.50pt -2\tabcolsep-1.5\arrayrulewidth}>{\centering\arraybackslash}p{\dimexpr 67.50pt -2\tabcolsep-1.5\arrayrulewidth}}
\toprule
 &  &  & \multicolumn{3}{>{\centering\arraybackslash}m{\dimexpr 202.50pt -2\tabcolsep-1.5\arrayrulewidth}}{V\textsubscript{\{ij\}}\textsuperscript{D}} \\ 
\cmidrule(lr){4-6}
 &  &  & f\textsubscript{1} (c\textsubscript{ij}) = 1/c\textsubscript{ij}\textsuperscript{3} & f\textsubscript{2} (c\textsubscript{ij}) = 1/c\textsubscript{ij}\textsuperscript{2} & f\textsubscript{3} (c\textsubscript{ij}) = 1/c\textsubscript{ij}\textsuperscript{0.1} \\ 
\cmidrule(lr){4-4} \cmidrule(lr){5-5} \cmidrule(lr){6-6}
Dest. & Population at 2 (units: \emph{people in 10,000s}) & Opportunities (units: \emph{capacity in 10,000s}) & units: \emph{physician-capacity in 10,000s} & units: \emph{physician-capacity in 10,000s} & units: \emph{physician-capacity in 10,000s} \\ 
\midrule\addlinespace[2.5pt]
1 & 10.000 & 7.000 & 2.133 & 2.272 & 3.411 \\ 
2 & 10.000 & 5.000 & 4.959 & 4.766 & 2.667 \\ 
3 & 10.000 & 8.000 & 2.904 & 2.958 & 3.919 \\ 
\bottomrule
\end{tabular*}

}

\end{table}%

Accessibility is conventionally presented as a zonal value, not a flow.
However, in the doubly-constrained case, since we force the allocation
of zonal population demand and zonal opportunities supplied to be paired
and allocation to be proportional, \(V_i^D\) is simply the number of
opportunities that matche our known population at \(i\). Following the
logic of the family of accessibility measures, in the doubly-constrained
case, \(V_{ij}^D\) flows are the only relevant unit of analysis: spatial
proportional allocations between population and opportunity capacity.
Furthermore, \(V^D_{ij}\) and its transposed counterpart \(M^D_{ji}\)
are structurally identical, differing only in interpretation (referring
to \(\kappa_{ij}^D\) and \(\hat \kappa_{ij}^D\)): one reflects the
proportional allocation of opportunity to population flows; the other,
population to opportunity flows.

\(V_{ij}^D\) are also mathematically equivalent to Wilson's spatial
interaction flows. And, as Wilson \citep{wilson1971} explicitly noted,
origin and destination weights defined in the spatial interaction model
\emph{can} be defined using any unit. However, the focus of these models
has typically been on \(ij\) flows, often calibrated using trips (i.e.,
outbound and inbound trips, inherently in the same units). With the
family of accessibility measures, it is made clear that we are working
in units of opportunities and population.

And as mentioned, these units are often misaligned. For instance, we may
not know how much park space, grocery area, or childcare capacity is
accessible per person. When they do align, such as people to physician
capacity, we would be modeling realized access flows based on known
quantities of \emph{those that interact} and the \emph{interacted}. In
such cases, the traditional accessibility question can be seen to be
already answered by the known information (i.e., how many opportunities
can be reached by a zone? Answer: the number of people at that zone).
For this reason, we do not foresee the doubly-constrained measure being
widely used in accessibility analysis, as the literature has largely
focused on questions of `potential', not on predicting flows of realized
access.

\section{8. Conclusions}\label{conclusions}

In this paper, we examined the historical and mathematical commonalities
between spatial interaction models and place-based accessibility
measures. As accessibility research evolved largely influenced by Hansen
\citep{hansen1959}, researchers in the field neglected the
proportionality constant that was originally present in gravity-based
models, and is still present in spatial interaction modeling. This work
has demonstrated theoretically, and through a simple numeric example,
that by reintroducing Wilson's system constraints and defining
associated balancing factors and proportional allocation factors, we can
derive a unifying family of accessibility measures that reintroduces
tangible units to the resulting values. These values may be more
intuitive for the purpose of analysis and comparison.

To summarize the contributions of this work, first, we place the popular
Hansen-type accessibility measure \citep{hansen1959} within this family
of measures as an ``unconstrained'' case, demonstrating that resulting
values cannot be directly compared across different travel scenarios
without ad-hoc adjustments. We then show how applying a total constraint
balances the units and produces a statistically averaged solution that
converges to the regional average for each zone as the decay effect
decreases. In other words, the total-constraint model could be a more
interpretable alternative for the unconstrained case if
population-competition is not relevant and one is interested in
capturing the maximum \emph{potential}; specifically, if there is a
fixed number of opportunities in the region, and if it makes sense to
assume that people accessing proximate opportunities leave fewer for
others, \emph{without} considering the population size at the origins.

We then introduced the singly-constrained case, which \emph{does} take
into account the population size at the origin in the allocation of
opportunities. It is also mathematically equivalent to the spatial
availability introduced in Soukhov et al.~2023
\citep{soukhovIntroducingSpatialAvailability2023}. In this case, all
accessibility values are fixed to sum to a known zonal opportunity-size
value (implicitly, the regional total of opportunities), but they are
not required to sum to any population-based values at the zone or
regional level. The singly-constrained model could be useful if regional
competition is a factor and if the acknowledgment that only a finite
number of opportunities can be allocated from each destination (with
those allocations distributed based on origin population size) is
suitable. We also introduce an `accessible' PPR (e.g., opportunities per
capita), calculated by dividing each accessibility value by the zonal
population. To clarify, this per capita expression of the
singly-constrained case is equivalent to the 2SFCA
\citep{luo2003, shen1998}, hence linking this literature back to spatial
interaction principles.

Lastly, the doubly-constrained case is introduced. In this case, the
sums must equal both the regional total and ensure that no zone
allocates more opportunities than it has available. Specifically,
accessibility values for each \(i\)-\(j\) pair must be a proportion of
the zonal opportunity and population values simultaneously. For example,
the accessibility at Zone 1 must equal the sum of opportunities from
Zones 1, 2, and 3, as well as the sum of the population at Zone 1.
Satisfying the double constraint means the opportunities and population
data must match one-to-one, so working with the accessibility
\(i\)-\(j\) pair value should be of research interest.

Building on Wilson's \citep{wilson1971} foundational work, this paper
proposes a unified framework for analyzing gravity-based accessibility.
By reintroducing Wilson's proportionality constants, the family of
constrained accessibility measures restores measurement units to
accessibility estimates. This enhancement provides a more interpretable,
consistent, and theoretically grounded basis for accessibility analysis,
which could help advance the adoption of accessibility-oriented
planning.

While this work exclusively focused on top-down gravity-based
accessibility measures, there have been recent developments in the
accessibility literature that include person-based approaches that are
time-sensitive \citep{yang2024evaluating, braga2023evaluating},
behavioural \citep{kar2024inclusive, lu2014effects} or utility-based
\citep{guzman2023much, ben1985discrete}. While gravity-based measures of
accessibility still dominate the applied literature, future work could
further explore how these ideas of proportionality constants and
balanced units could also help inform these and other new modeling
approaches.


\nolinenumbers
  \bibliography{bibliography.bib}


\end{document}
